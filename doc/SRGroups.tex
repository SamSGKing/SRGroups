% generated by GAPDoc2LaTeX from XML source (Frank Luebeck)
\documentclass[a4paper,11pt]{report}

\usepackage[top=37mm,bottom=37mm,left=27mm,right=27mm]{geometry}
\sloppy
\pagestyle{myheadings}
\usepackage{amssymb}
\usepackage[utf8]{inputenc}
\usepackage{makeidx}
\makeindex
\usepackage{color}
\definecolor{FireBrick}{rgb}{0.5812,0.0074,0.0083}
\definecolor{RoyalBlue}{rgb}{0.0236,0.0894,0.6179}
\definecolor{RoyalGreen}{rgb}{0.0236,0.6179,0.0894}
\definecolor{RoyalRed}{rgb}{0.6179,0.0236,0.0894}
\definecolor{LightBlue}{rgb}{0.8544,0.9511,1.0000}
\definecolor{Black}{rgb}{0.0,0.0,0.0}

\definecolor{linkColor}{rgb}{0.0,0.0,0.554}
\definecolor{citeColor}{rgb}{0.0,0.0,0.554}
\definecolor{fileColor}{rgb}{0.0,0.0,0.554}
\definecolor{urlColor}{rgb}{0.0,0.0,0.554}
\definecolor{promptColor}{rgb}{0.0,0.0,0.589}
\definecolor{brkpromptColor}{rgb}{0.589,0.0,0.0}
\definecolor{gapinputColor}{rgb}{0.589,0.0,0.0}
\definecolor{gapoutputColor}{rgb}{0.0,0.0,0.0}

%%  for a long time these were red and blue by default,
%%  now black, but keep variables to overwrite
\definecolor{FuncColor}{rgb}{0.0,0.0,0.0}
%% strange name because of pdflatex bug:
\definecolor{Chapter }{rgb}{0.0,0.0,0.0}
\definecolor{DarkOlive}{rgb}{0.1047,0.2412,0.0064}


\usepackage{fancyvrb}

\usepackage{mathptmx,helvet}
\usepackage[T1]{fontenc}
\usepackage{textcomp}


\usepackage[
            pdftex=true,
            bookmarks=true,        
            a4paper=true,
            pdftitle={Written with GAPDoc},
            pdfcreator={LaTeX with hyperref package / GAPDoc},
            colorlinks=true,
            backref=page,
            breaklinks=true,
            linkcolor=linkColor,
            citecolor=citeColor,
            filecolor=fileColor,
            urlcolor=urlColor,
            pdfpagemode={UseNone}, 
           ]{hyperref}

\newcommand{\maintitlesize}{\fontsize{50}{55}\selectfont}

% write page numbers to a .pnr log file for online help
\newwrite\pagenrlog
\immediate\openout\pagenrlog =\jobname.pnr
\immediate\write\pagenrlog{PAGENRS := [}
\newcommand{\logpage}[1]{\protect\write\pagenrlog{#1, \thepage,}}
%% were never documented, give conflicts with some additional packages

\newcommand{\GAP}{\textsf{GAP}}

%% nicer description environments, allows long labels
\usepackage{enumitem}
\setdescription{style=nextline}

%% depth of toc
\setcounter{tocdepth}{1}





%% command for ColorPrompt style examples
\newcommand{\gapprompt}[1]{\color{promptColor}{\bfseries #1}}
\newcommand{\gapbrkprompt}[1]{\color{brkpromptColor}{\bfseries #1}}
\newcommand{\gapinput}[1]{\color{gapinputColor}{#1}}


\begin{document}

\logpage{[ 0, 0, 0 ]}
\begin{titlepage}
\mbox{}\vfill

\begin{center}{\maintitlesize \textbf{ SRGroups \mbox{}}}\\
\vfill

\hypersetup{pdftitle= SRGroups }
\markright{\scriptsize \mbox{}\hfill  SRGroups  \hfill\mbox{}}
{\Huge \textbf{ Self-relicating groups acting on regular rooted trees \mbox{}}}\\
\vfill

{\Huge  0.9 \mbox{}}\\[1cm]
{ 6 July 2021 \mbox{}}\\[1cm]
\mbox{}\\[2cm]
{\Large \textbf{ Samuel King\\
   \mbox{}}}\\
{\Large \textbf{ Sarah Shotter\\
   \mbox{}}}\\
{\Large \textbf{ Stephan Tornier\\
    \mbox{}}}\\
\hypersetup{pdfauthor= Samuel King\\
   ;  Sarah Shotter\\
   ;  Stephan Tornier\\
    }
\end{center}\vfill

\mbox{}\\
{\mbox{}\\
\small \noindent \textbf{ Samuel King\\
   }  Email: \href{mailto://samuel.s.king@newcastle.edu.au} {\texttt{samuel.s.king@newcastle.edu.au}}\\
  Address: \begin{minipage}[t]{8cm}\noindent
 University Drive, Callaghan NSW 2308\\
 \end{minipage}
}\\
{\mbox{}\\
\small \noindent \textbf{ Sarah Shotter\\
   }  Email: \href{mailto://sarah.shotter@newcastle.edu.au} {\texttt{sarah.shotter@newcastle.edu.au}}\\
  Address: \begin{minipage}[t]{8cm}\noindent
 University Drive, Callaghan NSW 2308\\
 \end{minipage}
}\\
{\mbox{}\\
\small \noindent \textbf{ Stephan Tornier\\
    }  Email: \href{mailto://stephan.tornier@newcastle.edu.au} {\texttt{stephan.tornier@newcastle.edu.au}}\\
  Homepage: \href{https://www.newcastle.edu.au/profile/stephan-tornier} {\texttt{https://www.newcastle.edu.au/profile/stephan-tornier}}\\
  Address: \begin{minipage}[t]{8cm}\noindent
 University Drive, Callaghan NSW 2308\\
 \end{minipage}
}\\
\end{titlepage}

\newpage\setcounter{page}{2}
{\small 
\section*{Abstract}
\logpage{[ 0, 0, 1 ]}
 \textsf{SRGroups} is a package for searching up self-replicating groups of regular rooted trees
and performing computations on these groups. This package allows the user to
generate more self-replicating groups at greater depths with its in-built
functions, and is an extension of the \textsf{transgrp} package. \mbox{}}\\[1cm]
{\small 
\section*{Copyright}
\logpage{[ 0, 0, 2 ]}
 \textsf{SRGroups} is free software; you can redistribute it and/or modify it under the terms of
the \href{http://www.fsf.org/licenses/gpl.html} {GNU General Public License} as published by the Free Software Foundation; either version 3 of the License,
or (at your option) any later version. \mbox{}}\\[1cm]
{\small 
\section*{Acknowledgements}
\logpage{[ 0, 0, 3 ]}
 DE210100180, FL170100032. \mbox{}}\\[1cm]
\newpage

\def\contentsname{Contents\logpage{[ 0, 0, 4 ]}}

\tableofcontents
\newpage

     
\chapter{\textcolor{Chapter }{Introduction}}\label{Chapter_Introduction}
\logpage{[ 1, 0, 0 ]}
\hyperdef{L}{X7DFB63A97E67C0A1}{}
{
  Let $G$ be a subgroup of the regular rooted k-tree, $\textrm{Aut}(T_{k})$ with its group action, $\alpha$, defined as $\alpha(g,x)=g(x)$, where $g\in G$ are the automorphisms of $G$ and $x\in X$ the vertices of $T_{k}$. Let $\textrm{stab}_{G}(0)=\{g\in G : \alpha(g,0) = 0\}$, and $T_0\subset T_{k}$ be the set of all vertices below and including the vertex 0. Additonally, let $\varphi_0 : \textrm{stab}_G(0)\rightarrow G$ be a group homomorphism with the mapping $g\mapsto g|_{T_0}$. Then $G$ is called self-replicating if and only if the following two conditions, $\mathcal{R}_k$, are satisfied: $G$ is vertex transitive on level 1 of $T_{k}$, and $\varphi_0\left(\textrm{stab}_{G}(0)\right)=G$. 

 A group $G\leq\mathrm{Aut}(T_{k})$ acting greater than level 1 is said to have sufficient rigid automorphisms if
for each pair of vertices $u$ and $v$ on level $1$ of the tree, $T_{k,1}$, there exists an automorphism $g\in G$ such that $g(u)=v$ and $g|_u=e$, where $g$ is called $(u,v)$-rigid. For a self-replicating group $G$ on level $n$ of the tree, $T_{k,n}$, with sufficient rigid automorphisms, the maximal extension of $G$, $\mathcal{M}(G)$, is the largest self-replicating group (not necessarily with sufficient rigid
automorphisms) on $\textrm{Aut}(T_{k,n+1})$ that projects onto $G$, defined as: 
\[\mathcal{M}(G):= \{x\in\mathrm{Aut}(T_{k,n+1}) : \varphi_{n+1}(x)\in G and
x|_v\in G \textrm{for all v on level 1}\}.\]
 The self-replicating property is preserved across conjugacy. For a group $H\leq\mathrm{Aut}(T_{k,n})$ with sufficient rigid automorphisms, and a self-replicating group $G\leq\mathrm{Aut}(T_{k,n+1})$, there exists a conjugate of $G$ in $\mathrm{Aut}(T_{k,n+1})$ with sufficient rigid automorphisms. Since groups on level 1 inherently have
sufficient rigid automorphisms, then self-replicating groups with sufficient
rigid automorphisms can be found on all levels of the tree. 

 The \textsf{SRGroups} package serves to provide a library of these self-replicating groups to
further the ongoing studies of infinite networks and group theory. By using
the above definitions and conditions, several GAP methods and functions have
been built to allow computations of these groups and expand the understanding
of their behaviour. 

 First, this package acts as a library for searching currently known
self-replicating groups for varying degrees and levels of regular rooted
trees. This package also acts as a regular GAP package with functions that
allow the expansion of the library and addition of attributes/properties
relevant to self-replicating groups. Additional functions also exist in this
package that are compatible with GraphViz, to plot diagrams of the extension
behaviour of these self-replicating groups and their corresponding Hasse
diagrams at different depths. 
\section{\textcolor{Chapter }{Purpose}}\label{Chapter_Introduction_Section_Purpose}
\logpage{[ 1, 1, 0 ]}
\hyperdef{L}{X78E0A9867DABCE86}{}
{
  Research and educational purpose. To do. }

 }

   
\chapter{\textcolor{Chapter }{Preliminaries}}\label{Chapter_Preliminaries}
\logpage{[ 2, 0, 0 ]}
\hyperdef{L}{X8749E1888244CC3D}{}
{
  Introductory text. To do. 
\section{\textcolor{Chapter }{Regular rooted tree groups}}\label{Chapter_Preliminaries_Section_Regular_rooted_tree_groups}
\logpage{[ 2, 1, 0 ]}
\hyperdef{L}{X7B53F5737C0B7A9A}{}
{
  

\subsection{\textcolor{Chapter }{IsRegularRootedTreeGroup (for IsPermGroup)}}
\logpage{[ 2, 1, 1 ]}\nobreak
\hyperdef{L}{X81F8437A7C2E5C87}{}
{\noindent\textcolor{FuncColor}{$\triangleright$\enspace\texttt{IsRegularRootedTreeGroup({\mdseries\slshape arg})\index{IsRegularRootedTreeGroup@\texttt{IsRegularRootedTreeGroup}!for IsPermGroup}
\label{IsRegularRootedTreeGroup:for IsPermGroup}
}\hfill{\scriptsize (filter)}}\\
\textbf{\indent Returns:\ }
\texttt{true} or \texttt{false} 



 Groups acting on the regular rooted trees $T_{k,n}$ are stored together with their degree $k\in\mathbb{N}_{\ge 2}$ (see \texttt{RegularRootedTreeGroupDegree} (\ref{RegularRootedTreeGroupDegree})) and depth $n\in\mathbb{N}$ (see \texttt{RegularRootedTreeGroupDepth} (\ref{RegularRootedTreeGroupDepth})) as well as other attributes and properties in this category. 

 }

 

\subsection{\textcolor{Chapter }{RegularRootedTreeGroupDegree (for IsRegularRootedTreeGroup)}}
\logpage{[ 2, 1, 2 ]}\nobreak
\hyperdef{L}{X81620BBB8303B641}{}
{\noindent\textcolor{FuncColor}{$\triangleright$\enspace\texttt{RegularRootedTreeGroupDegree({\mdseries\slshape G})\index{RegularRootedTreeGroupDegree@\texttt{RegularRootedTreeGroupDegree}!for IsRegularRootedTreeGroup}
\label{RegularRootedTreeGroupDegree:for IsRegularRootedTreeGroup}
}\hfill{\scriptsize (attribute)}}\\
\textbf{\indent Returns:\ }
 The degree of \mbox{\texttt{\mdseries\slshape G}}. 



 The argument of this attribute is a regular rooted tree group \mbox{\texttt{\mdseries\slshape G}}. 

 }

 

 
\begin{Verbatim}[commandchars=!@|,fontsize=\small,frame=single,label=Example]
  !gapprompt@gap>| !gapinput@RegularRootedTreeGroupDegree(AutT(2,3));|
  2
\end{Verbatim}
 

\subsection{\textcolor{Chapter }{RegularRootedTreeGroupDepth (for IsRegularRootedTreeGroup)}}
\logpage{[ 2, 1, 3 ]}\nobreak
\hyperdef{L}{X7DA9D61678810B77}{}
{\noindent\textcolor{FuncColor}{$\triangleright$\enspace\texttt{RegularRootedTreeGroupDepth({\mdseries\slshape G})\index{RegularRootedTreeGroupDepth@\texttt{RegularRootedTreeGroupDepth}!for IsRegularRootedTreeGroup}
\label{RegularRootedTreeGroupDepth:for IsRegularRootedTreeGroup}
}\hfill{\scriptsize (attribute)}}\\
\textbf{\indent Returns:\ }
 The depth of \mbox{\texttt{\mdseries\slshape G}}. 



 The argument of this attribute is a regular rooted tree group \mbox{\texttt{\mdseries\slshape G}}. 

 }

 

 
\begin{Verbatim}[commandchars=!@|,fontsize=\small,frame=single,label=Example]
  !gapprompt@gap>| !gapinput@RegularRootedTreeGroupDepth(AutT(2,3));|
  3
\end{Verbatim}
 

\subsection{\textcolor{Chapter }{RegularRootedTreeGroup (for IsInt, IsInt, IsPermGroup)}}
\logpage{[ 2, 1, 4 ]}\nobreak
\hyperdef{L}{X7F279D957D29554A}{}
{\noindent\textcolor{FuncColor}{$\triangleright$\enspace\texttt{RegularRootedTreeGroup({\mdseries\slshape k, n, G})\index{RegularRootedTreeGroup@\texttt{RegularRootedTreeGroup}!for IsInt, IsInt, IsPermGroup}
\label{RegularRootedTreeGroup:for IsInt, IsInt, IsPermGroup}
}\hfill{\scriptsize (operation)}}\\
\textbf{\indent Returns:\ }
 The regular rooted tree group \mbox{\texttt{\mdseries\slshape G}} as an object of the category \texttt{IsRegularRootedTreeGroup} (\ref{IsRegularRootedTreeGroup}), together with its degree \mbox{\texttt{\mdseries\slshape k}} (see \texttt{RegularRootedTreeGroupDegree} (\ref{RegularRootedTreeGroupDegree})) and its depth \mbox{\texttt{\mdseries\slshape n}} (see \texttt{RegularRootedTreeGroupDepth} (\ref{RegularRootedTreeGroupDepth})). 



 The arguments of this method are a permutation group \mbox{\texttt{\mdseries\slshape G}} $\le\mathrm{Aut}(T_{k,n})$, its degree \mbox{\texttt{\mdseries\slshape k}} and its depth \mbox{\texttt{\mdseries\slshape n}}. 

 }

 }

 
\section{\textcolor{Chapter }{Auxiliary functions}}\label{Chapter_Preliminaries_Section_Auxiliary_functions}
\logpage{[ 2, 2, 0 ]}
\hyperdef{L}{X866E18057EF83F65}{}
{
  

\subsection{\textcolor{Chapter }{BelowAction}}
\logpage{[ 2, 2, 1 ]}\nobreak
\hyperdef{L}{X7EE8301C7EC8654B}{}
{\noindent\textcolor{FuncColor}{$\triangleright$\enspace\texttt{BelowAction({\mdseries\slshape k, n, aut, i})\index{BelowAction@\texttt{BelowAction}}
\label{BelowAction}
}\hfill{\scriptsize (function)}}\\
\textbf{\indent Returns:\ }
 The restriction of \mbox{\texttt{\mdseries\slshape aut}} to the subtree below the level 1 vertex \mbox{\texttt{\mdseries\slshape i}}, as an element of $\mathrm{Aut}(T_{k,n-1})$. 



 The arguments of this function are a degree \mbox{\texttt{\mdseries\slshape k}} $\in\mathbb{N}_{\ge 2}$, a depth \mbox{\texttt{\mdseries\slshape n}} $\in\mathbb{N}$, an element \mbox{\texttt{\mdseries\slshape aut}} of $\mathrm{Aut}(T_{k,n})$ (see \texttt{AutT} (\ref{AutT})), and a depth 1 vertex \mbox{\texttt{\mdseries\slshape i}} $\in\{1,\cdots,k\}$ of $T_{k,n}$. 

 }

 

 
\begin{Verbatim}[commandchars=!@|,fontsize=\small,frame=single,label=Example]
  !gapprompt@gap>| !gapinput@BelowAction(2,2,(1,2)(3,4),2);|
  (1,2)
\end{Verbatim}
 

\subsection{\textcolor{Chapter }{RemoveConjugates}}
\logpage{[ 2, 2, 2 ]}\nobreak
\hyperdef{L}{X795ACCF4858A8ED2}{}
{\noindent\textcolor{FuncColor}{$\triangleright$\enspace\texttt{RemoveConjugates({\mdseries\slshape G, subgroups})\index{RemoveConjugates@\texttt{RemoveConjugates}}
\label{RemoveConjugates}
}\hfill{\scriptsize (function)}}\\
\textbf{\indent Returns:\ }
 Removes \mbox{\texttt{\mdseries\slshape G}}-conjugates from the list \mbox{\texttt{\mdseries\slshape subgroups}}. 



 The arguments of this function are a group \mbox{\texttt{\mdseries\slshape G}} and a list \mbox{\texttt{\mdseries\slshape subgroups}} of subgroups of \mbox{\texttt{\mdseries\slshape G}}. 

 }

 

 
\begin{Verbatim}[commandchars=!@|,fontsize=\small,frame=single,label=Example]
  !gapprompt@gap>| !gapinput@G:=SymmetricGroup(3);|
  Sym( [ 1 .. 3 ] )
  !gapprompt@gap>| !gapinput@subgroups:=[Group((1,2)),Group((2,3)),Group((1,3))];|
  [ Group([ (1,2) ]), Group([ (2,3) ]), Group([ (1,3) ]) ]
  !gapprompt@gap>| !gapinput@RemoveConjugates(G,subgroups);|
  !gapprompt@gap>| !gapinput@subgroups;|
  [ Group([ (1,2) ]) ]
\end{Verbatim}
 }

 }

   
\chapter{\textcolor{Chapter }{Self-replicating groups}}\label{Chapter_Self-replicating_groups}
\logpage{[ 3, 0, 0 ]}
\hyperdef{L}{X876B0AE27E09225B}{}
{
  Introductory text. To do. 
\section{\textcolor{Chapter }{Properties and Attributes}}\label{Chapter_Self-replicating_groups_Section_Properties_and_Attributes}
\logpage{[ 3, 1, 0 ]}
\hyperdef{L}{X7A3E8C4478EBE1E7}{}
{
  

\subsection{\textcolor{Chapter }{IsSelfReplicating (for IsRegularRootedTreeGroup)}}
\logpage{[ 3, 1, 1 ]}\nobreak
\hyperdef{L}{X854CB0D683905369}{}
{\noindent\textcolor{FuncColor}{$\triangleright$\enspace\texttt{IsSelfReplicating({\mdseries\slshape G})\index{IsSelfReplicating@\texttt{IsSelfReplicating}!for IsRegularRootedTreeGroup}
\label{IsSelfReplicating:for IsRegularRootedTreeGroup}
}\hfill{\scriptsize (property)}}\\
\textbf{\indent Returns:\ }
 \texttt{true} if \mbox{\texttt{\mdseries\slshape G}} is self-replicating, and \texttt{false} otherwise. 



 The argument of this property is aa regular rooted tree group \mbox{\texttt{\mdseries\slshape G}}. 

 }

 

 
\begin{Verbatim}[commandchars=!@|,fontsize=\small,frame=single,label=Example]
  !gapprompt@gap>| !gapinput@subgroups:=AllSubgroups(AutT(2,2));|
  [ Group(()), Group([ (3,4) ]), Group([ (1,2) ]), Group([ (1,2)(3,4) ]), 
    Group([ (1,3)(2,4) ]), Group([ (1,4)(2,3) ]), Group([ (3,4), (1,2) ]), 
    Group([ (1,3)(2,4), (1,2)(3,4) ]), Group([ (1,3,2,4), (1,2)(3,4) ]), 
    Group([ (3,4), (1,2), (1,3)(2,4) ]) ]
  !gapprompt@gap>| !gapinput@Apply(subgroups,G->RegularRootedTreeGroup(2,2,G));|
  !gapprompt@gap>| !gapinput@Apply(subgroups,G->IsSelfReplicating(G));|
  !gapprompt@gap>| !gapinput@subgroups;|
  [ false, false, false, false, false, false, false, true, true, true ]
\end{Verbatim}
 

\subsection{\textcolor{Chapter }{HasSufficientRigidAutomorphisms (for IsRegularRootedTreeGroup)}}
\logpage{[ 3, 1, 2 ]}\nobreak
\hyperdef{L}{X84DDFFAE8701C3F8}{}
{\noindent\textcolor{FuncColor}{$\triangleright$\enspace\texttt{HasSufficientRigidAutomorphisms({\mdseries\slshape G})\index{HasSufficientRigidAutomorphisms@\texttt{HasSufficientRigidAutomorphisms}!for IsRegularRootedTreeGroup}
\label{HasSufficientRigidAutomorphisms:for IsRegularRootedTreeGroup}
}\hfill{\scriptsize (property)}}\\
\textbf{\indent Returns:\ }
 \texttt{true} if \mbox{\texttt{\mdseries\slshape G}} has sufficient rigid automorphisms, and \texttt{false} otherwise. 



 The argument of this property is a regular rooted tree group \mbox{\texttt{\mdseries\slshape G}} 

 }

 

 
\begin{Verbatim}[commandchars=!@|,fontsize=\small,frame=single,label=Example]
  !gapprompt@gap>| !gapinput@subgroups:=AllSubgroups(AutT(2,2));|
  [ Group(()), Group([ (3,4) ]), Group([ (1,2) ]), Group([ (1,2)(3,4) ]), 
    Group([ (1,3)(2,4) ]), Group([ (1,4)(2,3) ]), Group([ (3,4), (1,2) ]), 
    Group([ (1,3)(2,4), (1,2)(3,4) ]), Group([ (1,3,2,4), (1,2)(3,4) ]), 
    Group([ (3,4), (1,2), (1,3)(2,4) ]) ]
  !gapprompt@gap>| !gapinput@Apply(subgroups,G->RegularRootedTreeGroup(2,2,G));|
  !gapprompt@gap>| !gapinput@Apply(subgroups,G->HasSufficientRigidAutomorphisms(G));|
  !gapprompt@gap>| !gapinput@subgroups;|
  [ false, false, false, false, true, false, false, true, true, true ]
\end{Verbatim}
 

\subsection{\textcolor{Chapter }{ParentGroup (for IsRegularRootedTreeGroup)}}
\logpage{[ 3, 1, 3 ]}\nobreak
\hyperdef{L}{X7943CE9B7AB931AC}{}
{\noindent\textcolor{FuncColor}{$\triangleright$\enspace\texttt{ParentGroup({\mdseries\slshape G})\index{ParentGroup@\texttt{ParentGroup}!for IsRegularRootedTreeGroup}
\label{ParentGroup:for IsRegularRootedTreeGroup}
}\hfill{\scriptsize (attribute)}}\\
\textbf{\indent Returns:\ }
 The restriction of \mbox{\texttt{\mdseries\slshape G}} to $\mathrm{Aut}(T_{k,n-1})$. 



 The argument of this attribute is a regular rooted tree group \mbox{\texttt{\mdseries\slshape G}} $\le\mathrm{Aut}(T_{k,n})$. 

 }

 

 
\begin{Verbatim}[commandchars=!@|,fontsize=\small,frame=single,label=Example]
  !gapprompt@gap>| !gapinput@G:=AutT(2,3);;|
  !gapprompt@gap>| !gapinput@ParentGroup(G);|
  Group([ (1,2), (1,3)(2,4), (3,4) ])
  !gapprompt@gap>| !gapinput@last=AutT(2,2);|
  true
\end{Verbatim}
 

\subsection{\textcolor{Chapter }{MaximalExtension (for IsRegularRootedTreeGroup)}}
\logpage{[ 3, 1, 4 ]}\nobreak
\hyperdef{L}{X7D9FB6D5854A17BE}{}
{\noindent\textcolor{FuncColor}{$\triangleright$\enspace\texttt{MaximalExtension({\mdseries\slshape G})\index{MaximalExtension@\texttt{MaximalExtension}!for IsRegularRootedTreeGroup}
\label{MaximalExtension:for IsRegularRootedTreeGroup}
}\hfill{\scriptsize (attribute)}}\\
\textbf{\indent Returns:\ }
 The maximal extension of $M(G)\le\mathrm{Aut}(T_{k,n+1})$ of \mbox{\texttt{\mdseries\slshape G}}. 



 The argument of this attribute is a self-replicating regular rooted tree group \mbox{\texttt{\mdseries\slshape G}} $\le\mathrm{Aut}(T_{k,n})$ with sufficient rigid automorphisms. }

 

 
\begin{Verbatim}[commandchars=!@|,fontsize=\small,frame=single,label=Example]
  !gapprompt@gap>| !gapinput@G:=AutT(2,3);;|
  !gapprompt@gap>| !gapinput@MaximalExtension(G);|
  <permutation group with 11 generators>
  !gapprompt@gap>| !gapinput@last=AutT(2,4);|
  true
\end{Verbatim}
 

\subsection{\textcolor{Chapter }{RepresentativeWithSufficientRigidAutomorphisms (for IsRegularRootedTreeGroup)}}
\logpage{[ 3, 1, 5 ]}\nobreak
\hyperdef{L}{X7D06D7AF7B938F4A}{}
{\noindent\textcolor{FuncColor}{$\triangleright$\enspace\texttt{RepresentativeWithSufficientRigidAutomorphisms({\mdseries\slshape G})\index{RepresentativeWithSufficientRigidAutomorphisms@\texttt{Representative}\-\texttt{With}\-\texttt{Sufficient}\-\texttt{Rigid}\-\texttt{Automorphisms}!for IsRegularRootedTreeGroup}
\label{RepresentativeWithSufficientRigidAutomorphisms:for IsRegularRootedTreeGroup}
}\hfill{\scriptsize (attribute)}}\\
\textbf{\indent Returns:\ }
 A self-replicating $\mathrm{Aut}(T_{k,n})$-conjugate of \mbox{\texttt{\mdseries\slshape G}} with sufficient rigid automorphisms. If \mbox{\texttt{\mdseries\slshape G}} has sufficient rigid automorphisms then the output group has the same parent
group (see \texttt{ParentGroup} (\ref{ParentGroup})) as \mbox{\texttt{\mdseries\slshape G}}. 



 The argument of this attribute is a self-replicating regular rooted tree group \mbox{\texttt{\mdseries\slshape G}} $\le\mathrm{Aut}(T_{k,n})$. 

 }

 

 
\begin{Verbatim}[commandchars=!@|,fontsize=\small,frame=single,label=Example]
  !gapprompt@gap>| !gapinput@G:=SRGroup(2,3,6);|
  SRGroup(2,3,6)
  !gapprompt@gap>| !gapinput@conjugates:=ShallowCopy(AsList(G^AutT(2,3)));|
  [ Group([ (1,5)(2,6)(3,7)(4,8), (1,3)(2,4)(5,7)(6,8), (1,2)(3,4) ]), 
    Group([ (1,5)(2,6)(3,8)(4,7), (1,3)(2,4)(5,8)(6,7), (1,2)(3,4) ]) ]
  !gapprompt@gap>| !gapinput@Apply(conjugates,H->RegularRootedTreeGroup(2,3,H));|
  !gapprompt@gap>| !gapinput@for H in conjugates do Print(HasSufficientRigidAutomorphisms(H),"\n"); od;|
  true
  false
  !gapprompt@gap>| !gapinput@H:=conjugates[2];|
  Group([ (1,5)(2,6)(3,8)(4,7), (1,3)(2,4)(5,8)(6,7), (1,2)(3,4) ])
  !gapprompt@gap>| !gapinput@IsSelfReplicating(H);|
  true
  !gapprompt@gap>| !gapinput@RepresentativeWithSufficientRigidAutomorphisms(H);|
  Group([ (1,5)(2,6)(3,7)(4,8), (1,3)(2,4)(5,7)(6,8), (1,2)(3,4) ])
  !gapprompt@gap>| !gapinput@last=conjugates[1];|
  true
\end{Verbatim}
 }

 
\section{\textcolor{Chapter }{Examples}}\label{Chapter_Self-replicating_groups_Section_Examples}
\logpage{[ 3, 2, 0 ]}
\hyperdef{L}{X7A489A5D79DA9E5C}{}
{
  AutT. More to come. Grigorchuk, Hanoi, ... 

\subsection{\textcolor{Chapter }{AutT}}
\logpage{[ 3, 2, 1 ]}\nobreak
\hyperdef{L}{X843B26BC8517173F}{}
{\noindent\textcolor{FuncColor}{$\triangleright$\enspace\texttt{AutT({\mdseries\slshape k, n})\index{AutT@\texttt{AutT}}
\label{AutT}
}\hfill{\scriptsize (function)}}\\
\textbf{\indent Returns:\ }
 The regular rooted tree group $\mathrm{Aut}(T_{k,n})$ as a permutation group of the $k^{n}$ leaves of $T_{k,n}$. 



 The arguments of this function are a degree \mbox{\texttt{\mdseries\slshape k}} $\in\mathbb{N}_{\ge 2}$ and a depth \mbox{\texttt{\mdseries\slshape n}} $\in\mathbb{N}$. 

 }

 

 
\begin{Verbatim}[commandchars=!@|,fontsize=\small,frame=single,label=Example]
  !gapprompt@gap>| !gapinput@G:=AutT(2,2);|
  Group([ (1,2), (3,4), (1,3)(2,4) ])
  !gapprompt@gap>| !gapinput@Size(G);|
  8
\end{Verbatim}
 

\subsection{\textcolor{Chapter }{ConjugacyClassRepsMaxSelfReplicatingSubgroups}}
\logpage{[ 3, 2, 2 ]}\nobreak
\hyperdef{L}{X873DD3E87ADDC9D9}{}
{\noindent\textcolor{FuncColor}{$\triangleright$\enspace\texttt{ConjugacyClassRepsMaxSelfReplicatingSubgroups({\mdseries\slshape G})\index{ConjugacyClassRepsMaxSelfReplicatingSubgroups@\texttt{Conjugacy}\-\texttt{Class}\-\texttt{Reps}\-\texttt{Max}\-\texttt{Self}\-\texttt{Replicating}\-\texttt{Subgroups}}
\label{ConjugacyClassRepsMaxSelfReplicatingSubgroups}
}\hfill{\scriptsize (function)}}\\
\textbf{\indent Returns:\ }
 A list containing conjugacy class representatives of all maximal
self-replicating subgroups of \mbox{\texttt{\mdseries\slshape G}}. 



 The argument of this function is any regular rooted tree group, \mbox{\texttt{\mdseries\slshape G}} 

 }

 

 
\begin{Verbatim}[commandchars=!@|,fontsize=\small,frame=single,label=Example]
  !gapprompt@gap>| !gapinput@ConjugacyClassRepsMaxSelfReplicatingSubgroups(AutT(2,2));|
  [ Group([ (1,3)(2,4), (1,2)(3,4) ]), Group([ (1,3,2,4), (1,2)(3,4) ]) ]
\end{Verbatim}
 

\subsection{\textcolor{Chapter }{ConjugacyClassRepsSelfReplicatingSubgroupsWithConjugateProjection}}
\logpage{[ 3, 2, 3 ]}\nobreak
\hyperdef{L}{X7A0FFE297B512886}{}
{\noindent\textcolor{FuncColor}{$\triangleright$\enspace\texttt{ConjugacyClassRepsSelfReplicatingSubgroupsWithConjugateProjection({\mdseries\slshape G})\index{ConjugacyClassRepsSelfReplicatingSubgroupsWithConjugateProjection@\texttt{Conjugacy}\-\texttt{Class}\-\texttt{Reps}\-\texttt{Self}\-\texttt{Replicating}\-\texttt{Subgroups}\-\texttt{With}\-\texttt{Conjugate}\-\texttt{Projection}}
\label{ConjugacyClassRepsSelfReplicatingSubgroupsWithConjugateProjection}
}\hfill{\scriptsize (function)}}\\
\textbf{\indent Returns:\ }
 A list containing conjugacy class representatives of all self-replicating
subgroups of the maximal extension of \mbox{\texttt{\mdseries\slshape G}}, \mbox{\texttt{\mdseries\slshape M(G)}}. 



 The argument of this function is any regular rooted tree group, \mbox{\texttt{\mdseries\slshape G}} 

 }

 

 
\begin{Verbatim}[commandchars=!@|,fontsize=\small,frame=single,label=Example]
  !gapprompt@gap>| !gapinput@ConjugacyClassRepsSelfReplicatingSubgroupsWithConjugateProjection(AutT(3,1));|
  [ Group([ (1,4,7)(2,5,8)(3,6,9), (1,4)(2,5)(3,6), (1,2,3), (1,2) ]),
    Group([ (4,7)(5,8)(6,9), (1,4,7)(2,5,8)(3,6,9), (5,6)(8,9), (2,3)
        (8,9), (7,9,8), (4,6,5), (1,3,2) ]),
    Group([ (2,3)(4,7)(5,9)(6,8), (1,4,7)(2,5,8)(3,6,9), (5,6)(8,9),
        (2,3)(8,9), (7,9,8), (4,6,5), (1,3,2) ]),
    Group([ (2,3)(5,6)(8,9), (4,7)(5,8)(6,9), (1,4,7)(2,5,8)(3,6,9),
        (7,9,8), (4,6,5), (1,3,2) ]),
    Group([ (1,7)(2,8)(3,9)(5,6), (1,7,4)(2,9,5)(3,8,6), (1,2,3),
        (7,8,9), (4,6,5)(7,8,9) ]),
    Group([ (2,3)(4,7)(5,8)(6,9), (4,6,5)(7,9,8), (1,4,7)(2,5,9)
        (3,6,8), (1,2,3)(4,5,6)(7,9,8) ]),
    Group([ (2,3)(4,7)(5,8)(6,9), (1,7,6,2,9,4,3,8,5), (1,2,3)
        (4,6,5), (1,2,3)(4,5,6)(7,9,8) ]),
    Group([ (2,3)(4,7)(5,8)(6,9), (1,6,7,3,5,8,2,4,9), (1,3,2)(4,6,5)
        (7,8,9) ]),
    Group([ (2,3)(4,7)(5,8)(6,9), (1,7,4)(2,9,5)(3,8,6), (1,2,3)
        (4,5,6)(7,9,8) ]),
    Group([ (1,4)(2,5)(3,6), (1,7,4)(2,8,5)(3,9,6), (2,3)(5,6)(8,9),
        (1,2,3)(4,5,6)(7,8,9), (4,5,6)(7,9,8) ]),
    Group([ (2,3)(5,6)(8,9), (4,7)(5,8)(6,9), (1,4,7)(2,5,8)(3,6,9),
        (1,3,2)(4,6,5)(7,9,8) ]) ]
\end{Verbatim}
 }

 }

   
\chapter{\textcolor{Chapter }{The library of self-replicating groups}}\label{Chapter_The_library_of_self-replicating_groups}
\logpage{[ 4, 0, 0 ]}
\hyperdef{L}{X8261A1B883B61BC0}{}
{
  Introductory text. To do. 
\section{\textcolor{Chapter }{Availability functions}}\label{Chapter_The_library_of_self-replicating_groups_Section_Availability_functions}
\logpage{[ 4, 1, 0 ]}
\hyperdef{L}{X85E90392825E7312}{}
{
  Introductory text. To do. Similarities with transitive groups library. 

\subsection{\textcolor{Chapter }{SRGroupsAvailable}}
\logpage{[ 4, 1, 1 ]}\nobreak
\hyperdef{L}{X8500D0747EC1FCBF}{}
{\noindent\textcolor{FuncColor}{$\triangleright$\enspace\texttt{SRGroupsAvailable({\mdseries\slshape k, n})\index{SRGroupsAvailable@\texttt{SRGroupsAvailable}}
\label{SRGroupsAvailable}
}\hfill{\scriptsize (function)}}\\
\textbf{\indent Returns:\ }
 Whether the self-replicating groups of degree, \mbox{\texttt{\mdseries\slshape k}}, and depth, \mbox{\texttt{\mdseries\slshape n}}, are available. 



 The argument of this function is a degree, \mbox{\texttt{\mdseries\slshape k}}, and a depth, \mbox{\texttt{\mdseries\slshape n}}. }

 

 
\begin{Verbatim}[commandchars=!@|,fontsize=\small,frame=single,label=Example]
  !gapprompt@gap>| !gapinput@SRGroupsAvailable(2,5);|
  true
  !gapprompt@gap>| !gapinput@SRGroupsAvailable(5,2);|
  false
\end{Verbatim}
 

\subsection{\textcolor{Chapter }{NrSRGroups}}
\logpage{[ 4, 1, 2 ]}\nobreak
\hyperdef{L}{X81CEC3DC82CF99DE}{}
{\noindent\textcolor{FuncColor}{$\triangleright$\enspace\texttt{NrSRGroups({\mdseries\slshape k, n})\index{NrSRGroups@\texttt{NrSRGroups}}
\label{NrSRGroups}
}\hfill{\scriptsize (function)}}\\
\textbf{\indent Returns:\ }
 The number of self-replicating groups of degree, \mbox{\texttt{\mdseries\slshape k}}, and depth, \mbox{\texttt{\mdseries\slshape n}}. 



 The argument of this function is a degree, \mbox{\texttt{\mdseries\slshape k}}, and a depth, \mbox{\texttt{\mdseries\slshape n}}. }

 

 
\begin{Verbatim}[commandchars=!@|,fontsize=\small,frame=single,label=Example]
  !gapprompt@gap>| !gapinput@NrSRGroups(2,3);|
  15
  !gapprompt@gap>| !gapinput@NrSRGroups(2,5);|
  2436
\end{Verbatim}
 

\subsection{\textcolor{Chapter }{SRDegrees}}
\logpage{[ 4, 1, 3 ]}\nobreak
\hyperdef{L}{X7BF255357B870D37}{}
{\noindent\textcolor{FuncColor}{$\triangleright$\enspace\texttt{SRDegrees({\mdseries\slshape })\index{SRDegrees@\texttt{SRDegrees}}
\label{SRDegrees}
}\hfill{\scriptsize (function)}}\\
\textbf{\indent Returns:\ }
 All of the degrees currently stored in the \textsf{SRGroups} library. 



 There are no inputs to this function. }

 
\begin{Verbatim}[commandchars=!@|,fontsize=\small,frame=single,label=Example]
  !gapprompt@gap>| !gapinput@SRDegrees();|
  [ 2, 3, 4, 5, 6, 7, 8, 9, 10, 11, 12, 13, 14, 15, 16, 17 ]
\end{Verbatim}
 

\subsection{\textcolor{Chapter }{SRLevels}}
\logpage{[ 4, 1, 4 ]}\nobreak
\hyperdef{L}{X7CDBB45386329E21}{}
{\noindent\textcolor{FuncColor}{$\triangleright$\enspace\texttt{SRLevels({\mdseries\slshape k})\index{SRLevels@\texttt{SRLevels}}
\label{SRLevels}
}\hfill{\scriptsize (function)}}\\
\textbf{\indent Returns:\ }
 All of the levels currently stored in the \textsf{SRGroups} library for an input RegularRootedTreeGroupDegree, \mbox{\texttt{\mdseries\slshape deg}}. 



 The input to this function is the degree of the regular rooted tree, \mbox{\texttt{\mdseries\slshape k}}. }

 
\begin{Verbatim}[commandchars=!@|,fontsize=\small,frame=single,label=Example]
  !gapprompt@gap>| !gapinput@SRLevels(2);|
  [ 1, 2, 3, 4 ]
\end{Verbatim}
 }

 
\section{\textcolor{Chapter }{Selection functions}}\label{Chapter_The_library_of_self-replicating_groups_Section_Selection_functions}
\logpage{[ 4, 2, 0 ]}
\hyperdef{L}{X82676ED5826E9E2E}{}
{
  

\subsection{\textcolor{Chapter }{SRGroup}}
\logpage{[ 4, 2, 1 ]}\nobreak
\hyperdef{L}{X79B6921585FAFD67}{}
{\noindent\textcolor{FuncColor}{$\triangleright$\enspace\texttt{SRGroup({\mdseries\slshape k, n, num})\index{SRGroup@\texttt{SRGroup}}
\label{SRGroup}
}\hfill{\scriptsize (function)}}\\
\textbf{\indent Returns:\ }
 The \mbox{\texttt{\mdseries\slshape num}}th self-replicating group of degree \mbox{\texttt{\mdseries\slshape k}} and depth \mbox{\texttt{\mdseries\slshape n}} stored in the \textsf{SRGroups} library. 



 The argument of this function is a degree, \mbox{\texttt{\mdseries\slshape k}}, a depth, \mbox{\texttt{\mdseries\slshape n}}, and a designated number of the stored self-replicating group, \mbox{\texttt{\mdseries\slshape num}}. }

 

 
\begin{Verbatim}[commandchars=!@|,fontsize=\small,frame=single,label=Example]
  !gapprompt@gap>| !gapinput@SRGroup(2,3,1);|
  SRGroup(2,3,1)
  !gapprompt@gap>| !gapinput@Size(last);|
  8
\end{Verbatim}
 

\subsection{\textcolor{Chapter }{AllSRGroups}}
\logpage{[ 4, 2, 2 ]}\nobreak
\hyperdef{L}{X845F9F227B97062C}{}
{\noindent\textcolor{FuncColor}{$\triangleright$\enspace\texttt{AllSRGroups({\mdseries\slshape Input1, val1, Input2, val2, ...})\index{AllSRGroups@\texttt{AllSRGroups}}
\label{AllSRGroups}
}\hfill{\scriptsize (function)}}\\
\textbf{\indent Returns:\ }
 A list of self-replicating groups matching the input arguments as
RegularRootedTreeGroup objects. 



 Main library search function that acts analogously as the AllTransitiveGroups
function from the \textsf{transgrp} library. Has several possible input arguments such as \mbox{\texttt{\mdseries\slshape Degree}}, \mbox{\texttt{\mdseries\slshape Depth}} (or \mbox{\texttt{\mdseries\slshape Level}}), \mbox{\texttt{\mdseries\slshape Number}}, \mbox{\texttt{\mdseries\slshape Projection}}, \mbox{\texttt{\mdseries\slshape IsSubgroup}}, \mbox{\texttt{\mdseries\slshape Size}}, \mbox{\texttt{\mdseries\slshape NumberOfGenerators}}, and \mbox{\texttt{\mdseries\slshape IsAbelian}}. Order of the arguments do not matter. List inputs and singular inputs can be
provided. The argument definitions are as follows: \mbox{\texttt{\mdseries\slshape Degree}} (int {\textgreater} 1) := degree of tree \mbox{\texttt{\mdseries\slshape Depth}}/\mbox{\texttt{\mdseries\slshape Level}} (int {\textgreater} 0) := level of tree \mbox{\texttt{\mdseries\slshape Number}} (int {\textgreater} 0) := self-replicating group number \mbox{\texttt{\mdseries\slshape Projection}} (int {\textgreater} 0) := groups whose projected image are the group number on
the level above \mbox{\texttt{\mdseries\slshape IsSubgroup}} (int {\textgreater} 0) := groups that are a subgroup of the group number
provided \mbox{\texttt{\mdseries\slshape Size}} (int {\textgreater}= degree\texttt{\symbol{94}}depth or int {\textgreater} 1)
:= size of group \mbox{\texttt{\mdseries\slshape MinimalGeneratingSet}} (int {\textgreater} 0) := size of the group's minimal generating set \mbox{\texttt{\mdseries\slshape IsAbelian}} (boolean) := all groups that are abelian if true, and not abelian if false }

 
\begin{Verbatim}[commandchars=!@|,fontsize=\small,frame=single,label=Example]
  !gapprompt@gap>| !gapinput@AllSRGroups(Degree, 2, Level, 4, IsAbelian, true);|
  [ SRGroup(2,4,2), SRGroup(2,4,9), SRGroup(2,4,12), SRGroup(2,4,14) ]
  !gapprompt@gap>| !gapinput@AllSRGroups(Degree,[2..5],Depth,[2..5],IsSubgroup,[1..5],Projection,[1..3]);|
  Restricting degrees to [ 2, 3 ]
  [ SRGroup(2,1,1), SRGroup(2,1,1), SRGroup(2,2,1), SRGroup(2,3,1),
    SRGroup(2,3,2), SRGroup(2,4,1), SRGroup(2,4,1), SRGroup(2,4,2),
    SRGroup(2,4,2), SRGroup(2,4,2), SRGroup(3,1,1), SRGroup(3,1,1),
    SRGroup(3,1,1), SRGroup(3,1,1) ]
\end{Verbatim}
 

\subsection{\textcolor{Chapter }{AllSRGroupsInfo}}
\logpage{[ 4, 2, 3 ]}\nobreak
\hyperdef{L}{X808A9F918166020B}{}
{\noindent\textcolor{FuncColor}{$\triangleright$\enspace\texttt{AllSRGroupsInfo({\mdseries\slshape Input1, val1, Input2, val2, ...})\index{AllSRGroupsInfo@\texttt{AllSRGroupsInfo}}
\label{AllSRGroupsInfo}
}\hfill{\scriptsize (function)}}\\
\textbf{\indent Returns:\ }
 Information about the self-replicating group(s) satisfying all of the provided
input arguments in list form: [\mbox{\texttt{\mdseries\slshape Generators}}, \mbox{\texttt{\mdseries\slshape Name}}, \mbox{\texttt{\mdseries\slshape Parent Name}}, \mbox{\texttt{\mdseries\slshape Children Name(s)}}]. If the \mbox{\texttt{\mdseries\slshape Position}} input is provided, only the corresponding index of this list is returned. 



 Inputs work the same as the main library search function \texttt{AllSRGroups} (\ref{AllSRGroups}), with one additional input: \mbox{\texttt{\mdseries\slshape Position}} (or \mbox{\texttt{\mdseries\slshape Index}}). Position/Index := (int in [1..4]) }

 
\begin{Verbatim}[commandchars=!@|,fontsize=\small,frame=single,label=Example]
  !gapprompt@gap>| !gapinput@AllSRGroupsInfo(Degree, 2, Depth, [2,3], IsAbelian, true);|
  [ [ [ (1,2)(3,4), (1,3,2,4) ], "SRGroup(2,2,1)", "SRGroup(2,1,1)",
        [ "SRGroup(2,3,1)", "SRGroup(2,3,2)" ] ],
    [ [ (1,2)(3,4), (1,3)(2,4) ], "SRGroup(2,2,2)", "SRGroup(2,1,1)",
        [ "SRGroup(2,3,3)", "SRGroup(2,3,4)", "SRGroup(2,3,5)",
            "SRGroup(2,3,6)" ] ],
    [ [ (1,5,4,8,2,6,3,7), (1,4,2,3)(5,8,6,7), (1,2)(3,4)(5,6)(7,8) ]
          , "SRGroup(2,3,1)", "SRGroup(2,2,1)",
        [ "SRGroup(2,4,1)", "SRGroup(2,4,2)" ] ],
    [
        [ (1,5,2,6)(3,7,4,8), (1,3)(2,4)(5,7)(6,8),
            (1,2)(3,4)(5,6)(7,8) ], "SRGroup(2,3,4)",
        "SRGroup(2,2,2)",
        [ "SRGroup(2,4,8)", "SRGroup(2,4,9)", "SRGroup(2,4,10)" ] ],
    [ [ (1,3)(2,4)(5,7)(6,8), (1,5)(2,6)(3,7)(4,8),
            (1,2)(3,4)(5,6)(7,8) ], "SRGroup(2,3,5)",
        "SRGroup(2,2,2)",
        [ "SRGroup(2,4,11)", "SRGroup(2,4,12)", "SRGroup(2,4,13)",
            "SRGroup(2,4,14)", "SRGroup(2,4,15)" ] ] ]
  !gapprompt@gap>| !gapinput@AllSRGroupsInfo(Degree, 2, Level, [2,3], IsAbelian, true, Position, 1);|
  [ [ (1,2)(3,4), (1,3,2,4) ], [ (1,2)(3,4), (1,3)(2,4) ],
    [ (1,5,4,8,2,6,3,7), (1,4,2,3)(5,8,6,7), (1,2)(3,4)(5,6)(7,8) ],
    [ (1,5,2,6)(3,7,4,8), (1,3)(2,4)(5,7)(6,8), (1,2)(3,4)(5,6)(7,8) ],
    [ (1,3)(2,4)(5,7)(6,8), (1,5)(2,6)(3,7)(4,8), (1,2)(3,4)(5,6)(7,8) ] ]
\end{Verbatim}
 }

 
\section{\textcolor{Chapter }{Extending the library}}\label{Chapter_The_library_of_self-replicating_groups_Section_Extending_the_library}
\logpage{[ 4, 3, 0 ]}
\hyperdef{L}{X81DD090182ABC597}{}
{
  

\subsection{\textcolor{Chapter }{SRGroupFile}}
\logpage{[ 4, 3, 1 ]}\nobreak
\hyperdef{L}{X78CBB505845FA80F}{}
{\noindent\textcolor{FuncColor}{$\triangleright$\enspace\texttt{SRGroupFile({\mdseries\slshape k})\index{SRGroupFile@\texttt{SRGroupFile}}
\label{SRGroupFile}
}\hfill{\scriptsize (function)}}\\


 The arguments of this function are a degree, \mbox{\texttt{\mdseries\slshape k}}, or \mbox{\texttt{\mdseries\slshape 0}}. If the argument is non-zero, this function creates the file containing all
self-replicating groups of the regular rooted k-tree at the lowest level not
stored in the \textsf{SRGroups} library. If the argument is \mbox{\texttt{\mdseries\slshape 0}}, this function creates the file containing all self-replicating groups of the
regular rooted tree at the level 1 for the lowest degree not stored in the \textsf{SRGroups} library. The file naming convention is
"sr{\textunderscore}k{\textunderscore}n.grp", and they are stored in the
"data" folder of the \textsf{SRGroups} package. Level 1 groups are calculated using the \textsf{transgrp} library. If the argument is non-zero and there is a gap between files (i.e. if
"sr{\textunderscore}k{\textunderscore}n.grp" and
"sr{\textunderscore}k{\textunderscore}n+2.grp" exists, but
"sr{\textunderscore}k{\textunderscore}n+1.grp" does not exist), then this
function creates the files in this gap. 

 }

 

 
\begin{Verbatim}[commandchars=!@|,fontsize=\small,frame=single,label=Example]
  !gapprompt@gap>| !gapinput@SRGroupFile(2);|
  You have requested to make group files for degree 2.
  Creating level 3 file.
  Evaluating groups extending from:
  SRGroup(2,2,1)  (1/3)
  SRGroup(2,2,2)  (2/3)
  SRGroup(2,2,3)  (3/3)
  SRGroup(2,2,4)  (4/3)
  Formatting file sr_2_3.grp now.
  Reordering individual files.
  Done.
  !gapprompt@gap>| !gapinput@SRGroupFile(0);|
  Creating degree 5 file on level 1.
  Done.
  !gapprompt@gap>| !gapinput@SRGroupFile(2);|
  You have requested to make group files for degree 2.
  Gap found; missing file from level 2. Creating the missing file now.
  Creating files:
  sr_2_2.grp
  Done.
\end{Verbatim}
 

\subsection{\textcolor{Chapter }{ExtendSRGroup}}
\logpage{[ 4, 3, 2 ]}\nobreak
\hyperdef{L}{X8307E6897847A11B}{}
{\noindent\textcolor{FuncColor}{$\triangleright$\enspace\texttt{ExtendSRGroup({\mdseries\slshape arg})\index{ExtendSRGroup@\texttt{ExtendSRGroup}}
\label{ExtendSRGroup}
}\hfill{\scriptsize (function)}}\\


 The arguments of this function are: arg[1]: degree of tree (int {\textgreater}
1), \mbox{\texttt{\mdseries\slshape k}}, arg[2]: highest level of tree where the file
"sr{\textunderscore}k{\textunderscore}n.grp" exists (int {\textgreater} 1), \mbox{\texttt{\mdseries\slshape n}}, (arg[3],arg[4],...): sequence of group numbers to extend from using the
files
"temp{\textunderscore}k{\textunderscore}n{\textunderscore}arg[3]{\textunderscore}arg[4]{\textunderscore}...arg[Length(arg)-1].grp".
This function creates the file of the group number arg[Length(arg)] stored in
the file
"temp{\textunderscore}k{\textunderscore}n{\textunderscore}arg[3]{\textunderscore}arg[4]{\textunderscore}...arg[Length(arg)-1].grp",
and saves it as
"temp{\textunderscore}k{\textunderscore}n{\textunderscore}arg[3]{\textunderscore}arg[4]{\textunderscore}...arg[Length(arg)].grp". 

 }

 

\subsection{\textcolor{Chapter }{CombineSRFiles}}
\logpage{[ 4, 3, 3 ]}\nobreak
\hyperdef{L}{X78CB3D847AA20727}{}
{\noindent\textcolor{FuncColor}{$\triangleright$\enspace\texttt{CombineSRFiles({\mdseries\slshape k, n})\index{CombineSRFiles@\texttt{CombineSRFiles}}
\label{CombineSRFiles}
}\hfill{\scriptsize (function)}}\\


 The arguments of this function are a degree, \mbox{\texttt{\mdseries\slshape k}}, and a level, \mbox{\texttt{\mdseries\slshape n}}, of a regular rooted tree, \mbox{\texttt{\mdseries\slshape n-1}} is the highest level stored as the file
"sr{\textunderscore}k{\textunderscore}n-1.grp" in the \textsf{SRGroups} library, and all of the files
"temp{\textunderscore}k{\textunderscore}n-1{\textunderscore}i{\textunderscore}proj.grp"
for every SRGroup(k,n-1,i) are stored in the
"data/temp{\textunderscore}k{\textunderscore}n" folder of the \textsf{SRGroups} library. This function combines each of the
"temp{\textunderscore}k{\textunderscore}n-1{\textunderscore}i{\textunderscore}proj.grp"
files into the complete "temp{\textunderscore}k{\textunderscore}n.grp" file to
be used by the \texttt{SRGroupFile} (\ref{SRGroupFile}) function. 

 }

 

\subsection{\textcolor{Chapter }{CheckSRProjections}}
\logpage{[ 4, 3, 4 ]}\nobreak
\hyperdef{L}{X7DC9A60E87E350AF}{}
{\noindent\textcolor{FuncColor}{$\triangleright$\enspace\texttt{CheckSRProjections({\mdseries\slshape k, n})\index{CheckSRProjections@\texttt{CheckSRProjections}}
\label{CheckSRProjections}
}\hfill{\scriptsize (function)}}\\
\textbf{\indent Returns:\ }
 Whether all of the self-replicating groups of degree \mbox{\texttt{\mdseries\slshape k}} and level \mbox{\texttt{\mdseries\slshape n}} project correctly to level \mbox{\texttt{\mdseries\slshape n-1}}. This is mainly used after obtaining new data to check that it has been
formatted correctly (see \texttt{SRGroupFile} (\ref{SRGroupFile})). 



 The arguments of this function are a degree, \mbox{\texttt{\mdseries\slshape k}}, and a level, \mbox{\texttt{\mdseries\slshape n}}. }

 
\begin{Verbatim}[commandchars=!@|,fontsize=\small,frame=single,label=Example]
  !gapprompt@gap>| !gapinput@CheckSRProjections(2,4);|
  All groups project correctly.
\end{Verbatim}
 }

 }

 \def\indexname{Index\logpage{[ "Ind", 0, 0 ]}
\hyperdef{L}{X83A0356F839C696F}{}
}

\cleardoublepage
\phantomsection
\addcontentsline{toc}{chapter}{Index}


\printindex

\immediate\write\pagenrlog{["Ind", 0, 0], \arabic{page},}
\immediate\write\pagenrlog{["Ind", 0, 0], \arabic{page},}
\immediate\write\pagenrlog{["Ind", 0, 0], \arabic{page},}
\immediate\write\pagenrlog{["Ind", 0, 0], \arabic{page},}
\immediate\write\pagenrlog{["Ind", 0, 0], \arabic{page},}
\immediate\write\pagenrlog{["Ind", 0, 0], \arabic{page},}
\immediate\write\pagenrlog{["Ind", 0, 0], \arabic{page},}
\immediate\write\pagenrlog{["Ind", 0, 0], \arabic{page},}
\immediate\write\pagenrlog{["Ind", 0, 0], \arabic{page},}
\immediate\write\pagenrlog{["Ind", 0, 0], \arabic{page},}
\immediate\write\pagenrlog{["Ind", 0, 0], \arabic{page},}
\immediate\write\pagenrlog{["Ind", 0, 0], \arabic{page},}
\immediate\write\pagenrlog{["Ind", 0, 0], \arabic{page},}
\immediate\write\pagenrlog{["Ind", 0, 0], \arabic{page},}
\immediate\write\pagenrlog{["Ind", 0, 0], \arabic{page},}
\immediate\write\pagenrlog{["Ind", 0, 0], \arabic{page},}
\immediate\write\pagenrlog{["Ind", 0, 0], \arabic{page},}
\newpage
\immediate\write\pagenrlog{["End"], \arabic{page}];}
\immediate\closeout\pagenrlog
\end{document}
