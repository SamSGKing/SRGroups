% generated by GAPDoc2LaTeX from XML source (Frank Luebeck)
\documentclass[a4paper,11pt]{report}

\usepackage[top=37mm,bottom=37mm,left=27mm,right=27mm]{geometry}
\sloppy
\pagestyle{myheadings}
\usepackage{amssymb}
\usepackage[utf8]{inputenc}
\usepackage{makeidx}
\makeindex
\usepackage{color}
\definecolor{FireBrick}{rgb}{0.5812,0.0074,0.0083}
\definecolor{RoyalBlue}{rgb}{0.0236,0.0894,0.6179}
\definecolor{RoyalGreen}{rgb}{0.0236,0.6179,0.0894}
\definecolor{RoyalRed}{rgb}{0.6179,0.0236,0.0894}
\definecolor{LightBlue}{rgb}{0.8544,0.9511,1.0000}
\definecolor{Black}{rgb}{0.0,0.0,0.0}

\definecolor{linkColor}{rgb}{0.0,0.0,0.554}
\definecolor{citeColor}{rgb}{0.0,0.0,0.554}
\definecolor{fileColor}{rgb}{0.0,0.0,0.554}
\definecolor{urlColor}{rgb}{0.0,0.0,0.554}
\definecolor{promptColor}{rgb}{0.0,0.0,0.589}
\definecolor{brkpromptColor}{rgb}{0.589,0.0,0.0}
\definecolor{gapinputColor}{rgb}{0.589,0.0,0.0}
\definecolor{gapoutputColor}{rgb}{0.0,0.0,0.0}

%%  for a long time these were red and blue by default,
%%  now black, but keep variables to overwrite
\definecolor{FuncColor}{rgb}{0.0,0.0,0.0}
%% strange name because of pdflatex bug:
\definecolor{Chapter }{rgb}{0.0,0.0,0.0}
\definecolor{DarkOlive}{rgb}{0.1047,0.2412,0.0064}


\usepackage{fancyvrb}

\usepackage{mathptmx,helvet}
\usepackage[T1]{fontenc}
\usepackage{textcomp}


\usepackage[
            pdftex=true,
            bookmarks=true,        
            a4paper=true,
            pdftitle={Written with GAPDoc},
            pdfcreator={LaTeX with hyperref package / GAPDoc},
            colorlinks=true,
            backref=page,
            breaklinks=true,
            linkcolor=linkColor,
            citecolor=citeColor,
            filecolor=fileColor,
            urlcolor=urlColor,
            pdfpagemode={UseNone}, 
           ]{hyperref}

\newcommand{\maintitlesize}{\fontsize{50}{55}\selectfont}

% write page numbers to a .pnr log file for online help
\newwrite\pagenrlog
\immediate\openout\pagenrlog =\jobname.pnr
\immediate\write\pagenrlog{PAGENRS := [}
\newcommand{\logpage}[1]{\protect\write\pagenrlog{#1, \thepage,}}
%% were never documented, give conflicts with some additional packages

\newcommand{\GAP}{\textsf{GAP}}

%% nicer description environments, allows long labels
\usepackage{enumitem}
\setdescription{style=nextline}

%% depth of toc
\setcounter{tocdepth}{1}





%% command for ColorPrompt style examples
\newcommand{\gapprompt}[1]{\color{promptColor}{\bfseries #1}}
\newcommand{\gapbrkprompt}[1]{\color{brkpromptColor}{\bfseries #1}}
\newcommand{\gapinput}[1]{\color{gapinputColor}{#1}}


\begin{document}

\logpage{[ 0, 0, 0 ]}
\begin{titlepage}
\mbox{}\vfill

\begin{center}{\maintitlesize \textbf{ SRGroups: Self-replicating groups of regular rooted trees. \mbox{}}}\\
\vfill

\hypersetup{pdftitle= SRGroups: Self-replicating groups of regular rooted trees. }
\markright{\scriptsize \mbox{}\hfill  SRGroups: Self-replicating groups of regular rooted trees.  \hfill\mbox{}}
{\Huge \textbf{ Self-replicating groups of regular rooted trees. \mbox{}}}\\
\vfill

{\Huge  0.1 \mbox{}}\\[1cm]
{ 17 December 2020 \mbox{}}\\[1cm]
\mbox{}\\[2cm]
{\Large \textbf{ Sam King\\
   \mbox{}}}\\
{\Large \textbf{ Sarah Shotter\\
   \mbox{}}}\\
\hypersetup{pdfauthor= Sam King\\
   ;  Sarah Shotter\\
   }
\end{center}\vfill

\mbox{}\\
{\mbox{}\\
\small \noindent \textbf{ Sam King\\
   }  Email: \href{mailto://sam.king@newcastle.edu.au} {\texttt{sam.king@newcastle.edu.au}}\\
  Address: \begin{minipage}[t]{8cm}\noindent
 University Drive, Callaghan NSW 2308\\
 \end{minipage}
}\\
{\mbox{}\\
\small \noindent \textbf{ Sarah Shotter\\
   }  Email: \href{mailto://sarah.shotter@newcastle.edu.au} {\texttt{sarah.shotter@newcastle.edu.au}}\\
  Address: \begin{minipage}[t]{8cm}\noindent
 University Drive, Callaghan NSW 2308\\
 \end{minipage}
}\\
\end{titlepage}

\newpage\setcounter{page}{2}
{\small 
\section*{Abstract}
\logpage{[ 0, 0, 1 ]}
 \textsf{SRGroups} is a package for searching up self-replicating groups of regular rooted trees
and performing computations on these groups. This package allows the user to
generate more self-replicating groups at greater depths with its in-built
functions, and is an extension of the \textsf{transgrp} package. \mbox{}}\\[1cm]
{\small 
\section*{Copyright}
\logpage{[ 0, 0, 2 ]}
 \textsf{SRGroups} is free software; you can redistribute it and/or modify it under the terms of
the \href{http://www.fsf.org/licenses/gpl.html} {GNU General Public License} as published by the Free Software Foundation; either version 3 of the License,
or (at your option) any later version. \mbox{}}\\[1cm]
{\small 
\section*{Acknowledgements}
\logpage{[ 0, 0, 3 ]}
 DE210100180, FL170100032. \mbox{}}\\[1cm]
\newpage

\def\contentsname{Contents\logpage{[ 0, 0, 4 ]}}

\tableofcontents
\newpage

     
\chapter{\textcolor{Chapter }{Introduction}}\label{Chapter_Introduction}
\logpage{[ 1, 0, 0 ]}
\hyperdef{L}{X7DFB63A97E67C0A1}{}
{
  SRGroups is a package which does some interesting and cool things. To be
continued... }

   
\chapter{\textcolor{Chapter }{Functionality}}\label{Chapter_Functionality}
\logpage{[ 2, 0, 0 ]}
\hyperdef{L}{X87F1120883F5B4D0}{}
{
  
\section{\textcolor{Chapter }{Methods}}\label{Chapter_Functionality_Section_Methods}
\logpage{[ 2, 1, 0 ]}
\hyperdef{L}{X8606FDCE878850EF}{}
{
  

\subsection{\textcolor{Chapter }{IsRegularRootedTreeGroup (for IsPermGroup)}}
\logpage{[ 2, 1, 1 ]}\nobreak
\hyperdef{L}{X81F8437A7C2E5C87}{}
{\noindent\textcolor{FuncColor}{$\triangleright$\enspace\texttt{IsRegularRootedTreeGroup({\mdseries\slshape G})\index{IsRegularRootedTreeGroup@\texttt{IsRegularRootedTreeGroup}!for IsPermGroup}
\label{IsRegularRootedTreeGroup:for IsPermGroup}
}\hfill{\scriptsize (filter)}}\\
\textbf{\indent Returns:\ }
\texttt{true} or \texttt{false} 



 The argument of this category is any permutation group, \mbox{\texttt{\mdseries\slshape G}}. Checks whether \mbox{\texttt{\mdseries\slshape G}} is a regular rooted tree group. }

 

\subsection{\textcolor{Chapter }{RegularRootedTreeGroupDegree (for IsRegularRootedTreeGroup)}}
\logpage{[ 2, 1, 2 ]}\nobreak
\hyperdef{L}{X81620BBB8303B641}{}
{\noindent\textcolor{FuncColor}{$\triangleright$\enspace\texttt{RegularRootedTreeGroupDegree({\mdseries\slshape G})\index{RegularRootedTreeGroupDegree@\texttt{RegularRootedTreeGroupDegree}!for IsRegularRootedTreeGroup}
\label{RegularRootedTreeGroupDegree:for IsRegularRootedTreeGroup}
}\hfill{\scriptsize (attribute)}}\\
\textbf{\indent Returns:\ }
 The degree of \mbox{\texttt{\mdseries\slshape G}}. 



 The argument of this attribute is any regular rooted tree group, \mbox{\texttt{\mdseries\slshape G}}. }

 
\begin{Verbatim}[commandchars=!@|,fontsize=\small,frame=single,label=Example]
  !gapprompt@gap>| !gapinput@RegularRootedTreeGroupDepth(AutT(2,3));|
  3
\end{Verbatim}
 

\subsection{\textcolor{Chapter }{RegularRootedTreeGroupDepth (for IsRegularRootedTreeGroup)}}
\logpage{[ 2, 1, 3 ]}\nobreak
\hyperdef{L}{X7DA9D61678810B77}{}
{\noindent\textcolor{FuncColor}{$\triangleright$\enspace\texttt{RegularRootedTreeGroupDepth({\mdseries\slshape G})\index{RegularRootedTreeGroupDepth@\texttt{RegularRootedTreeGroupDepth}!for IsRegularRootedTreeGroup}
\label{RegularRootedTreeGroupDepth:for IsRegularRootedTreeGroup}
}\hfill{\scriptsize (attribute)}}\\
\textbf{\indent Returns:\ }
 The depth of \mbox{\texttt{\mdseries\slshape G}}. 



 The argument of this attribute is any regular rooted tree group, \mbox{\texttt{\mdseries\slshape G}}. }

 
\begin{Verbatim}[commandchars=!@|,fontsize=\small,frame=single,label=Example]
  !gapprompt@gap>| !gapinput@RegularRootedTreeGroupDegree(AutT(2,3));|
  2
\end{Verbatim}
 

\subsection{\textcolor{Chapter }{RegularRootedTreeGroup (for IsInt, IsInt, IsPermGroup)}}
\logpage{[ 2, 1, 4 ]}\nobreak
\hyperdef{L}{X7F279D957D29554A}{}
{\noindent\textcolor{FuncColor}{$\triangleright$\enspace\texttt{RegularRootedTreeGroup({\mdseries\slshape k, n, G})\index{RegularRootedTreeGroup@\texttt{RegularRootedTreeGroup}!for IsInt, IsInt, IsPermGroup}
\label{RegularRootedTreeGroup:for IsInt, IsInt, IsPermGroup}
}\hfill{\scriptsize (operation)}}\\
\textbf{\indent Returns:\ }
 The regular rooted tree group \mbox{\texttt{\mdseries\slshape G}} as an object of the category \texttt{IsRegularRootedTreeGroup} (\ref{IsRegularRootedTreeGroup}), with attributes \texttt{RegularRootedTreeGroupDegree} (\ref{RegularRootedTreeGroupDegree}) and \texttt{RegularRootedTreeGroupDepth} (\ref{RegularRootedTreeGroupDepth}). 



 The arguments of this operation are a regular rooted tree group, \mbox{\texttt{\mdseries\slshape G}}, and its degree \mbox{\texttt{\mdseries\slshape k}} and depth \mbox{\texttt{\mdseries\slshape n}}. }

 

\subsection{\textcolor{Chapter }{IsSelfReplicating (for IsRegularRootedTreeGroup)}}
\logpage{[ 2, 1, 5 ]}\nobreak
\hyperdef{L}{X854CB0D683905369}{}
{\noindent\textcolor{FuncColor}{$\triangleright$\enspace\texttt{IsSelfReplicating({\mdseries\slshape G})\index{IsSelfReplicating@\texttt{IsSelfReplicating}!for IsRegularRootedTreeGroup}
\label{IsSelfReplicating:for IsRegularRootedTreeGroup}
}\hfill{\scriptsize (property)}}\\
\textbf{\indent Returns:\ }
\texttt{true} or \texttt{false} 



 The argument of this property is any regular rooted tree group, \mbox{\texttt{\mdseries\slshape G}}. Tests whether \mbox{\texttt{\mdseries\slshape G}} satisfies the self-replicating conditions. 

 }

 

 
\begin{Verbatim}[commandchars=!@|,fontsize=\small,frame=single,label=Example]
  !gapprompt@gap>| !gapinput@IsSelfReplicating(AutT(2,3));|
  true
\end{Verbatim}
 

\subsection{\textcolor{Chapter }{HasSufficientRigidAutomorphisms (for IsRegularRootedTreeGroup)}}
\logpage{[ 2, 1, 6 ]}\nobreak
\hyperdef{L}{X84DDFFAE8701C3F8}{}
{\noindent\textcolor{FuncColor}{$\triangleright$\enspace\texttt{HasSufficientRigidAutomorphisms({\mdseries\slshape G})\index{HasSufficientRigidAutomorphisms@\texttt{HasSufficientRigidAutomorphisms}!for IsRegularRootedTreeGroup}
\label{HasSufficientRigidAutomorphisms:for IsRegularRootedTreeGroup}
}\hfill{\scriptsize (property)}}\\
\textbf{\indent Returns:\ }
\texttt{true} or \texttt{false} 



 The argument of this property is any regular rooted tree group, \mbox{\texttt{\mdseries\slshape G}}. Tests whether \mbox{\texttt{\mdseries\slshape G}} has sufficient rigid automorphisms. 

 }

 

 
\begin{Verbatim}[commandchars=!@|,fontsize=\small,frame=single,label=Example]
  !gapprompt@gap>| !gapinput@HasSufficientRigidAutomorphisms(AutT(2,3));|
  true
\end{Verbatim}
 

\subsection{\textcolor{Chapter }{ParentGroup (for IsRegularRootedTreeGroup)}}
\logpage{[ 2, 1, 7 ]}\nobreak
\hyperdef{L}{X7943CE9B7AB931AC}{}
{\noindent\textcolor{FuncColor}{$\triangleright$\enspace\texttt{ParentGroup({\mdseries\slshape G})\index{ParentGroup@\texttt{ParentGroup}!for IsRegularRootedTreeGroup}
\label{ParentGroup:for IsRegularRootedTreeGroup}
}\hfill{\scriptsize (attribute)}}\\
\textbf{\indent Returns:\ }
 The image of \mbox{\texttt{\mdseries\slshape G}} when projected onto the automorphism group of degree \mbox{\texttt{\mdseries\slshape k}} and depth \mbox{\texttt{\mdseries\slshape n-1}}. 



 The argument of this attribute is any regular rooted tree group, \mbox{\texttt{\mdseries\slshape G}}, of degree \mbox{\texttt{\mdseries\slshape k}} and depth \mbox{\texttt{\mdseries\slshape n}}. }

 

 
\begin{Verbatim}[commandchars=!@|,fontsize=\small,frame=single,label=Example]
  !gapprompt@gap>| !gapinput@G:=AutT(2,3); H:=AutT(2,2);|
  Group([ (1,2), (3,4), (5,6), (7,8), (1,3)(2,4), (5,7)(6,8), (1,5)(2,6)(3,7)(4,8) ])
  Group([ (1,2), (3,4), (1,3)(2,4) ])
  !gapprompt@gap>| !gapinput@ParentGroup(G);|
  Group([ (1,2), (1,3)(2,4), (3,4) ])
  !gapprompt@gap>| !gapinput@H=last;|
  true
\end{Verbatim}
 

\subsection{\textcolor{Chapter }{MaximalExtension (for IsRegularRootedTreeGroup)}}
\logpage{[ 2, 1, 8 ]}\nobreak
\hyperdef{L}{X7D9FB6D5854A17BE}{}
{\noindent\textcolor{FuncColor}{$\triangleright$\enspace\texttt{MaximalExtension({\mdseries\slshape G})\index{MaximalExtension@\texttt{MaximalExtension}!for IsRegularRootedTreeGroup}
\label{MaximalExtension:for IsRegularRootedTreeGroup}
}\hfill{\scriptsize (attribute)}}\\
\textbf{\indent Returns:\ }
 The maximal extension of \mbox{\texttt{\mdseries\slshape G}}, \mbox{\texttt{\mdseries\slshape M(G)}}, that is a subgroup of the automorphism group of degree \mbox{\texttt{\mdseries\slshape k}} and depth \mbox{\texttt{\mdseries\slshape n+1}}. 



 The argument of this attribute is any regular rooted tree group, \mbox{\texttt{\mdseries\slshape G}}, of degree \mbox{\texttt{\mdseries\slshape k}} and depth \mbox{\texttt{\mdseries\slshape n}}. }

 

 
\begin{Verbatim}[commandchars=!@|,fontsize=\small,frame=single,label=Example]
  !gapprompt@gap>| !gapinput@G:=AutT(2,3); H:=AutT(2,4);|
  Group([ (1,2), (3,4), (5,6), (7,8), (1,3)(2,4), (5,7)(6,8), (1,5)(2,6)(3,7)(4,8) ])
  <permutation group of size 32768 with 15 generators>
  !gapprompt@gap>| !gapinput@MaximalExtension(G);|
  <permutation group with 11 generators>
  !gapprompt@gap>| !gapinput@H=last;|
  true
\end{Verbatim}
 

\subsection{\textcolor{Chapter }{RepresentativeWithSufficientRigidAutomorphisms (for IsRegularRootedTreeGroup)}}
\logpage{[ 2, 1, 9 ]}\nobreak
\hyperdef{L}{X7D06D7AF7B938F4A}{}
{\noindent\textcolor{FuncColor}{$\triangleright$\enspace\texttt{RepresentativeWithSufficientRigidAutomorphisms({\mdseries\slshape G})\index{RepresentativeWithSufficientRigidAutomorphisms@\texttt{Representative}\-\texttt{With}\-\texttt{Sufficient}\-\texttt{Rigid}\-\texttt{Automorphisms}!for IsRegularRootedTreeGroup}
\label{RepresentativeWithSufficientRigidAutomorphisms:for IsRegularRootedTreeGroup}
}\hfill{\scriptsize (attribute)}}\\
\textbf{\indent Returns:\ }
 A conjugate of \mbox{\texttt{\mdseries\slshape G}} with sufficient rigid automorphisms. 



 The argument of this attribute is any regular rooted tree group, \mbox{\texttt{\mdseries\slshape G}}. }

 

 
\begin{Verbatim}[commandchars=!@|,fontsize=\small,frame=single,label=Example]
  gap>
\end{Verbatim}
 }

 
\section{\textcolor{Chapter }{Library Functions}}\label{Chapter_Functionality_Section_Library_Functions}
\logpage{[ 2, 2, 0 ]}
\hyperdef{L}{X84F966937C2E128A}{}
{
  

\subsection{\textcolor{Chapter }{AllSRGroups}}
\logpage{[ 2, 2, 1 ]}\nobreak
\hyperdef{L}{X845F9F227B97062C}{}
{\noindent\textcolor{FuncColor}{$\triangleright$\enspace\texttt{AllSRGroups({\mdseries\slshape Input1, val1, Input2, val2, ...})\index{AllSRGroups@\texttt{AllSRGroups}}
\label{AllSRGroups}
}\hfill{\scriptsize (function)}}\\
\textbf{\indent Returns:\ }
 All of the self-replicating group(s) stored as objects satisfying all of the
provided input arguments. 



 Main library search function. Has several possible input arguments such as \mbox{\texttt{\mdseries\slshape Degree}}, \mbox{\texttt{\mdseries\slshape Level}} (or \mbox{\texttt{\mdseries\slshape Depth}}), \mbox{\texttt{\mdseries\slshape Number}}, \mbox{\texttt{\mdseries\slshape Projection}}, \mbox{\texttt{\mdseries\slshape Subgroup}}, \mbox{\texttt{\mdseries\slshape Size}}, \mbox{\texttt{\mdseries\slshape NumberOfGenerators}}, and \mbox{\texttt{\mdseries\slshape IsAbelian}}. Order of the inputs do not matter. }

 
\begin{Verbatim}[commandchars=!@|,fontsize=\small,frame=single,label=Example]
  !gapprompt@gap>| !gapinput@AllSRGroups(Degree, 2, Level, 4, IsAbelian, true);|
  [ SRGroup(2,4,2), SRGroup(2,4,9), SRGroup(2,4,12), SRGroup(2,4,14) ]
  !gapprompt@gap>| !gapinput@Size(last[1]);|
  16
  !gapprompt@gap>| !gapinput@AllSRGroups(Degree, 2, Level, 4, NumberOfGenerators, 4);|
  [ SRGroup(2,4,11), SRGroup(2,4,12), SRGroup(2,4,16), SRGroup(2,4,20), SRGroup(2,4,23), SRGroup(2,4,24),
   SRGroup(2,4,25), SRGroup(2,4,26), SRGroup(2,4,40), SRGroup(2,4,43), SRGroup(2,4,46), SRGroup(2,4,47),
   SRGroup(2,4,50), SRGroup(2,4,66), SRGroup(2,4,70), SRGroup(2,4,71), SRGroup(2,4,72), SRGroup(2,4,73),
   SRGroup(2,4,74), SRGroup(2,4,75), SRGroup(2,4,76), SRGroup(2,4,84), SRGroup(2,4,90), SRGroup(2,4,91),
   SRGroup(2,4,93), SRGroup(2,4,95), SRGroup(2,4,97), SRGroup(2,4,102), SRGroup(2,4,108) ]
\end{Verbatim}
 

\subsection{\textcolor{Chapter }{AllSRGroupsInfo}}
\logpage{[ 2, 2, 2 ]}\nobreak
\hyperdef{L}{X808A9F918166020B}{}
{\noindent\textcolor{FuncColor}{$\triangleright$\enspace\texttt{AllSRGroupsInfo({\mdseries\slshape Input1, val1, Input2, val2, ...})\index{AllSRGroupsInfo@\texttt{AllSRGroupsInfo}}
\label{AllSRGroupsInfo}
}\hfill{\scriptsize (function)}}\\
\textbf{\indent Returns:\ }
 Information about the self-replicating group(s) satisfying all of the provided
input arguments in list form: [\mbox{\texttt{\mdseries\slshape Generators}}, \mbox{\texttt{\mdseries\slshape Name}}, \mbox{\texttt{\mdseries\slshape Parent Name}}, \mbox{\texttt{\mdseries\slshape Children Name(s)}}]. If the \mbox{\texttt{\mdseries\slshape Position}} input is provided, only the corresponding index of this list is returned. 



 Inputs work the same as the main library search function \texttt{AllSRGroups} (\ref{AllSRGroups}), with one additional input: \mbox{\texttt{\mdseries\slshape Position}}. }

 
\begin{Verbatim}[commandchars=!@|,fontsize=\small,frame=single,label=Example]
  !gapprompt@gap>| !gapinput@AllSRGroupsInfo(Degree, 2, Level, 3, IsAbelian, true);|
  [ [ [ (1,5,4,8,2,6,3,7), (1,4,2,3)(5,8,6,7), (1,2)(3,4)(5,6)(7,8) ], "SRGroup(2,3,1)", "SRGroup(2,2,1)", [ "SRGroup(2,4,1)", "SRGroup(2,4,2)" ] ],
  [ [ (1,5,2,6)(3,7,4,8), (1,3)(2,4)(5,7)(6,8), (1,2)(3,4)(5,6)(7,8) ], "SRGroup(2,3,4)", "SRGroup(2,2,2)", [ "SRGroup(2,4,8)", "SRGroup(2,4,9)", "SRGroup(2,4,10)" ] ], 
  [ [ (1,3)(2,4)(5,7)(6,8), (1,5)(2,6)(3,7)(4,8), (1,2)(3,4)(5,6)(7,8) ], "SRGroup(2,3,5)", "SRGroup(2,2,2)", [ "SRGroup(2,4,11)", "SRGroup(2,4,12)", "SRGroup(2,4,13)", "SRGroup(2,4,14)", "SRGroup(2,4,15)" ] ] ]
  !gapprompt@gap>| !gapinput@AllSRGroupsInfo(Degree, 2, Level, 3, IsAbelian, true, Position, 1);|
  [ [ (1,5,4,8,2,6,3,7), (1,4,2,3)(5,8,6,7), (1,2)(3,4)(5,6)(7,8) ],
    [ (1,5,2,6)(3,7,4,8), (1,3)(2,4)(5,7)(6,8), (1,2)(3,4)(5,6)(7,8) ],
    [ (1,3)(2,4)(5,7)(6,8), (1,5)(2,6)(3,7)(4,8), (1,2)(3,4)(5,6)(7,8) ] ]
\end{Verbatim}
 

\subsection{\textcolor{Chapter }{SRDegrees}}
\logpage{[ 2, 2, 3 ]}\nobreak
\hyperdef{L}{X7BF255357B870D37}{}
{\noindent\textcolor{FuncColor}{$\triangleright$\enspace\texttt{SRDegrees({\mdseries\slshape })\index{SRDegrees@\texttt{SRDegrees}}
\label{SRDegrees}
}\hfill{\scriptsize (function)}}\\
\textbf{\indent Returns:\ }
 All of the degrees currently stored in the \textsf{SRGroups} library (duplicates included). 



 There are no inputs to this function. }

 
\begin{Verbatim}[commandchars=!@|,fontsize=\small,frame=single,label=Example]
  !gapprompt@gap>| !gapinput@SRDegrees();|
  [ 2, 2, 2, 2, 3, 3, 3, 4, 4, 5, 6, 7, 8, 9, 10, 11, 12, 13, 14, 15, 16 ]
\end{Verbatim}
 

\subsection{\textcolor{Chapter }{SRLevels}}
\logpage{[ 2, 2, 4 ]}\nobreak
\hyperdef{L}{X7CDBB45386329E21}{}
{\noindent\textcolor{FuncColor}{$\triangleright$\enspace\texttt{SRLevels({\mdseries\slshape k})\index{SRLevels@\texttt{SRLevels}}
\label{SRLevels}
}\hfill{\scriptsize (function)}}\\
\textbf{\indent Returns:\ }
 All of the levels currently stored in the \textsf{SRGroups} library for an input RegularRootedTreeGroupDegree, \mbox{\texttt{\mdseries\slshape deg}}. 



 Degree of regular rooted tree, \mbox{\texttt{\mdseries\slshape k}}. }

 
\begin{Verbatim}[commandchars=!@|,fontsize=\small,frame=single,label=Example]
  !gapprompt@gap>| !gapinput@SRLevels(2);|
  [ 1, 2, 3, 4 ]
\end{Verbatim}
 }

 
\section{\textcolor{Chapter }{Package Functions}}\label{Chapter_Functionality_Section_Package_Functions}
\logpage{[ 2, 3, 0 ]}
\hyperdef{L}{X7FF84D377C18FE95}{}
{
  

\subsection{\textcolor{Chapter }{AutT}}
\logpage{[ 2, 3, 1 ]}\nobreak
\hyperdef{L}{X843B26BC8517173F}{}
{\noindent\textcolor{FuncColor}{$\triangleright$\enspace\texttt{AutT({\mdseries\slshape k, n})\index{AutT@\texttt{AutT}}
\label{AutT}
}\hfill{\scriptsize (function)}}\\
\textbf{\indent Returns:\ }
 The regular rooted tree group $\mathrm{Aut}(T_{k,n})$ as a permutation group of the $k^{n}$ leaves of $T_{k,n}$. 



 The arguments of this function are a degree \mbox{\texttt{\mdseries\slshape k}} $\in\mathbb{N}_{\ge 2}$ and a depth \mbox{\texttt{\mdseries\slshape n}} $\in\mathbb{N}$. 

 }

 

 
\begin{Verbatim}[commandchars=!@|,fontsize=\small,frame=single,label=Example]
  !gapprompt@gap>| !gapinput@G:=AutT(2,2);|
  Group([ (1,2), (3,4), (1,3)(2,4) ])
  !gapprompt@gap>| !gapinput@Size(G);|
  8
\end{Verbatim}
 

\subsection{\textcolor{Chapter }{BelowAction}}
\logpage{[ 2, 3, 2 ]}\nobreak
\hyperdef{L}{X7EE8301C7EC8654B}{}
{\noindent\textcolor{FuncColor}{$\triangleright$\enspace\texttt{BelowAction({\mdseries\slshape k, n, aut, i})\index{BelowAction@\texttt{BelowAction}}
\label{BelowAction}
}\hfill{\scriptsize (function)}}\\
\textbf{\indent Returns:\ }
 The restriction of \mbox{\texttt{\mdseries\slshape aut}} to the subtree below the level 1 vertex \mbox{\texttt{\mdseries\slshape i}}, as an element of \texttt{AutT(}\mbox{\texttt{\mdseries\slshape k}},\mbox{\texttt{\mdseries\slshape n-1}}\texttt{)}. 



 The arguments of this function are a degree, \mbox{\texttt{\mdseries\slshape k}} $\in\mathbb{N}_{\ge 2}$, a depth, \mbox{\texttt{\mdseries\slshape n}} $\in\mathbb{N}$, an element of \texttt{AutT(}\mbox{\texttt{\mdseries\slshape k}},\mbox{\texttt{\mdseries\slshape n}}\texttt{)}, \mbox{\texttt{\mdseries\slshape aut}}, and a level 1 vertex, \mbox{\texttt{\mdseries\slshape i}} $\in\{1,\cdots,k\}$. 

 }

 

 
\begin{Verbatim}[commandchars=!@|,fontsize=\small,frame=single,label=Example]
  !gapprompt@gap>| !gapinput@BelowAction(2,2,(1,2)(3,4),2);|
  (1,2)
\end{Verbatim}
 }

 }

 \def\indexname{Index\logpage{[ "Ind", 0, 0 ]}
\hyperdef{L}{X83A0356F839C696F}{}
}

\cleardoublepage
\phantomsection
\addcontentsline{toc}{chapter}{Index}


\printindex

\immediate\write\pagenrlog{["Ind", 0, 0], \arabic{page},}
\immediate\write\pagenrlog{["Ind", 0, 0], \arabic{page},}
\immediate\write\pagenrlog{["Ind", 0, 0], \arabic{page},}
\immediate\write\pagenrlog{["Ind", 0, 0], \arabic{page},}
\immediate\write\pagenrlog{["Ind", 0, 0], \arabic{page},}
\immediate\write\pagenrlog{["Ind", 0, 0], \arabic{page},}
\immediate\write\pagenrlog{["Ind", 0, 0], \arabic{page},}
\immediate\write\pagenrlog{["Ind", 0, 0], \arabic{page},}
\immediate\write\pagenrlog{["Ind", 0, 0], \arabic{page},}
\immediate\write\pagenrlog{["Ind", 0, 0], \arabic{page},}
\immediate\write\pagenrlog{["Ind", 0, 0], \arabic{page},}
\immediate\write\pagenrlog{["Ind", 0, 0], \arabic{page},}
\immediate\write\pagenrlog{["Ind", 0, 0], \arabic{page},}
\immediate\write\pagenrlog{["Ind", 0, 0], \arabic{page},}
\immediate\write\pagenrlog{["Ind", 0, 0], \arabic{page},}
\newpage
\immediate\write\pagenrlog{["End"], \arabic{page}];}
\immediate\closeout\pagenrlog
\end{document}
