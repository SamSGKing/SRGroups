% generated by GAPDoc2LaTeX from XML source (Frank Luebeck)
\documentclass[a4paper,11pt]{report}

\usepackage[top=37mm,bottom=37mm,left=27mm,right=27mm]{geometry}
\sloppy
\pagestyle{myheadings}
\usepackage{amssymb}
\usepackage[utf8]{inputenc}
\usepackage{makeidx}
\makeindex
\usepackage{color}
\definecolor{FireBrick}{rgb}{0.5812,0.0074,0.0083}
\definecolor{RoyalBlue}{rgb}{0.0236,0.0894,0.6179}
\definecolor{RoyalGreen}{rgb}{0.0236,0.6179,0.0894}
\definecolor{RoyalRed}{rgb}{0.6179,0.0236,0.0894}
\definecolor{LightBlue}{rgb}{0.8544,0.9511,1.0000}
\definecolor{Black}{rgb}{0.0,0.0,0.0}

\definecolor{linkColor}{rgb}{0.0,0.0,0.554}
\definecolor{citeColor}{rgb}{0.0,0.0,0.554}
\definecolor{fileColor}{rgb}{0.0,0.0,0.554}
\definecolor{urlColor}{rgb}{0.0,0.0,0.554}
\definecolor{promptColor}{rgb}{0.0,0.0,0.589}
\definecolor{brkpromptColor}{rgb}{0.589,0.0,0.0}
\definecolor{gapinputColor}{rgb}{0.589,0.0,0.0}
\definecolor{gapoutputColor}{rgb}{0.0,0.0,0.0}

%%  for a long time these were red and blue by default,
%%  now black, but keep variables to overwrite
\definecolor{FuncColor}{rgb}{0.0,0.0,0.0}
%% strange name because of pdflatex bug:
\definecolor{Chapter }{rgb}{0.0,0.0,0.0}
\definecolor{DarkOlive}{rgb}{0.1047,0.2412,0.0064}


\usepackage{fancyvrb}

\usepackage{mathptmx,helvet}
\usepackage[T1]{fontenc}
\usepackage{textcomp}


\usepackage[
            pdftex=true,
            bookmarks=true,        
            a4paper=true,
            pdftitle={Written with GAPDoc},
            pdfcreator={LaTeX with hyperref package / GAPDoc},
            colorlinks=true,
            backref=page,
            breaklinks=true,
            linkcolor=linkColor,
            citecolor=citeColor,
            filecolor=fileColor,
            urlcolor=urlColor,
            pdfpagemode={UseNone}, 
           ]{hyperref}

\newcommand{\maintitlesize}{\fontsize{50}{55}\selectfont}

% write page numbers to a .pnr log file for online help
\newwrite\pagenrlog
\immediate\openout\pagenrlog =\jobname.pnr
\immediate\write\pagenrlog{PAGENRS := [}
\newcommand{\logpage}[1]{\protect\write\pagenrlog{#1, \thepage,}}
%% were never documented, give conflicts with some additional packages

\newcommand{\GAP}{\textsf{GAP}}

%% nicer description environments, allows long labels
\usepackage{enumitem}
\setdescription{style=nextline}

%% depth of toc
\setcounter{tocdepth}{1}





%% command for ColorPrompt style examples
\newcommand{\gapprompt}[1]{\color{promptColor}{\bfseries #1}}
\newcommand{\gapbrkprompt}[1]{\color{brkpromptColor}{\bfseries #1}}
\newcommand{\gapinput}[1]{\color{gapinputColor}{#1}}


\begin{document}

\logpage{[ 0, 0, 0 ]}
\begin{titlepage}
\mbox{}\vfill

\begin{center}{\maintitlesize \textbf{ SRGroups \mbox{}}}\\
\vfill

\hypersetup{pdftitle= SRGroups }
\markright{\scriptsize \mbox{}\hfill  SRGroups  \hfill\mbox{}}
{\Huge \textbf{ Self-replicating groups of regular rooted trees. \mbox{}}}\\
\vfill

{\Huge  0.1 \mbox{}}\\[1cm]
{ 17 December 2020 \mbox{}}\\[1cm]
\mbox{}\\[2cm]
{\Large \textbf{ Sam King\\
   \mbox{}}}\\
{\Large \textbf{ Sarah Shotter\\
   \mbox{}}}\\
\hypersetup{pdfauthor= Sam King\\
   ;  Sarah Shotter\\
   }
\end{center}\vfill

\mbox{}\\
{\mbox{}\\
\small \noindent \textbf{ Sam King\\
   }  Email: \href{mailto://sam.king@newcastle.edu.au} {\texttt{sam.king@newcastle.edu.au}}\\
  Address: \begin{minipage}[t]{8cm}\noindent
 University Drive, Callaghan NSW 2308\\
 \end{minipage}
}\\
{\mbox{}\\
\small \noindent \textbf{ Sarah Shotter\\
   }  Email: \href{mailto://sarah.shotter@newcastle.edu.au} {\texttt{sarah.shotter@newcastle.edu.au}}\\
  Address: \begin{minipage}[t]{8cm}\noindent
 University Drive, Callaghan NSW 2308\\
 \end{minipage}
}\\
\end{titlepage}

\newpage\setcounter{page}{2}
{\small 
\section*{Abstract}
\logpage{[ 0, 0, 1 ]}
 To do. \mbox{}}\\[1cm]
{\small 
\section*{Copyright}
\logpage{[ 0, 0, 2 ]}
 \textsf{???} is free software; you can redistribute it and/or modify it under the terms of
the \href{http://www.fsf.org/licenses/gpl.html} {GNU General Public License} as published by the Free Software Foundation; either version 3 of the License,
or (at your option) any later version. \mbox{}}\\[1cm]
{\small 
\section*{Acknowledgements}
\logpage{[ 0, 0, 3 ]}
 DE210100180, FL170100032. \mbox{}}\\[1cm]
\newpage

\def\contentsname{Contents\logpage{[ 0, 0, 4 ]}}

\tableofcontents
\newpage

     
\chapter{\textcolor{Chapter }{The package}}\label{Chapter_The_package}
\logpage{[ 1, 0, 0 ]}
\hyperdef{L}{X79DE59997FDCF767}{}
{
  ??? is a package which does some interesting and cool things. To be
continued... 
\section{\textcolor{Chapter }{Framework}}\label{Chapter_The_package_Section_Framework}
\logpage{[ 1, 1, 0 ]}
\hyperdef{L}{X7B17FC3E7F9C9985}{}
{
  

\subsection{\textcolor{Chapter }{IsRegularRootedTreeGroup (for IsPermGroup)}}
\logpage{[ 1, 1, 1 ]}\nobreak
\hyperdef{L}{X81F8437A7C2E5C87}{}
{\noindent\textcolor{FuncColor}{$\triangleright$\enspace\texttt{IsRegularRootedTreeGroup({\mdseries\slshape arg})\index{IsRegularRootedTreeGroup@\texttt{IsRegularRootedTreeGroup}!for IsPermGroup}
\label{IsRegularRootedTreeGroup:for IsPermGroup}
}\hfill{\scriptsize (filter)}}\\
\textbf{\indent Returns:\ }
\texttt{true} or \texttt{false} 



 Groups acting on regular rooted trees are stored together with their degree (\texttt{RegularRootedTreeGroupDegree} (\ref{RegularRootedTreeGroupDegree})), depth (\texttt{RegularRootedTreeGroupDepth} (\ref{RegularRootedTreeGroupDepth})) and other attributes in this category. See also \texttt{RegularRootedTreeGroup} (\ref{RegularRootedTreeGroup}). See also \texttt{AutT} (\ref{AutT}). 

 }

 
\begin{Verbatim}[commandchars=!@|,fontsize=\small,frame=single,label=Example]
  !gapprompt@gap>| !gapinput@G:=SymmetricGroup(3);|
  Sym( [ 1 .. 3 ] )
  !gapprompt@gap>| !gapinput@IsRegularRootedTreeGroup(G);|
  false
  !gapprompt@gap>| !gapinput@H:=RegularRootedTreeGroup(3,1,SymmetricGroup(3));|
  Sym( [ 1 .. 3 ] )
  !gapprompt@gap>| !gapinput@IsRegularRootedTreeGroup(H);|
  true
\end{Verbatim}
 

\subsection{\textcolor{Chapter }{RegularRootedTreeGroup (for IsInt, IsInt, IsPermGroup)}}
\logpage{[ 1, 1, 2 ]}\nobreak
\hyperdef{L}{X7F279D957D29554A}{}
{\noindent\textcolor{FuncColor}{$\triangleright$\enspace\texttt{RegularRootedTreeGroup({\mdseries\slshape k, n, G})\index{RegularRootedTreeGroup@\texttt{RegularRootedTreeGroup}!for IsInt, IsInt, IsPermGroup}
\label{RegularRootedTreeGroup:for IsInt, IsInt, IsPermGroup}
}\hfill{\scriptsize (operation)}}\\
\textbf{\indent Returns:\ }
 the regular rooted tree group $G$ as an object of the category \texttt{IsRegularRootedTreeGroup} (\ref{IsRegularRootedTreeGroup}). 



 The arguments of this method are a degree \mbox{\texttt{\mdseries\slshape k}} $\in\mathbb{N}_{\ge 2}$, a depth \mbox{\texttt{\mdseries\slshape n}} $\in\mathbb{N}$ and a subgroup \mbox{\texttt{\mdseries\slshape G}} of $\mathrm{Aut}(T_{k,n})$. 

 }

 
\begin{Verbatim}[commandchars=!@|,fontsize=\small,frame=single,label=Example]
  to do
\end{Verbatim}
 

\subsection{\textcolor{Chapter }{RegularRootedTreeGroupDegree (for IsRegularRootedTreeGroup)}}
\logpage{[ 1, 1, 3 ]}\nobreak
\hyperdef{L}{X81620BBB8303B641}{}
{\noindent\textcolor{FuncColor}{$\triangleright$\enspace\texttt{RegularRootedTreeGroupDegree({\mdseries\slshape G})\index{RegularRootedTreeGroupDegree@\texttt{RegularRootedTreeGroupDegree}!for IsRegularRootedTreeGroup}
\label{RegularRootedTreeGroupDegree:for IsRegularRootedTreeGroup}
}\hfill{\scriptsize (attribute)}}\\
\textbf{\indent Returns:\ }
 the degree \mbox{\texttt{\mdseries\slshape k}} of the regular rooted tree that \mbox{\texttt{\mdseries\slshape G}} is acting on. 



 The argument of this attribute is a regular rooted tree group \mbox{\texttt{\mdseries\slshape G}} $\le\mathrm{Aut}(T_{k,k})$ (\texttt{IsRegularRootedTreeGroup} (\ref{IsRegularRootedTreeGroup})). 

 }

 

 
\begin{Verbatim}[commandchars=!@|,fontsize=\small,frame=single,label=Example]
  to do
\end{Verbatim}
 

\subsection{\textcolor{Chapter }{RegularRootedTreeGroupDepth (for IsRegularRootedTreeGroup)}}
\logpage{[ 1, 1, 4 ]}\nobreak
\hyperdef{L}{X7DA9D61678810B77}{}
{\noindent\textcolor{FuncColor}{$\triangleright$\enspace\texttt{RegularRootedTreeGroupDepth({\mdseries\slshape G})\index{RegularRootedTreeGroupDepth@\texttt{RegularRootedTreeGroupDepth}!for IsRegularRootedTreeGroup}
\label{RegularRootedTreeGroupDepth:for IsRegularRootedTreeGroup}
}\hfill{\scriptsize (attribute)}}\\
\textbf{\indent Returns:\ }
 the depth \mbox{\texttt{\mdseries\slshape n}} of the regular rooted tree that \mbox{\texttt{\mdseries\slshape G}} is acting on. 



 The argument of this attribute is a regular rooted tree group \mbox{\texttt{\mdseries\slshape G}} $\le\mathrm{Aut}(T_{k,k})$ (\texttt{IsRegularRootedTreeGroup} (\ref{IsRegularRootedTreeGroup})). 

 }

 

 
\begin{Verbatim}[commandchars=!@|,fontsize=\small,frame=single,label=Example]
  to do
\end{Verbatim}
 

\subsection{\textcolor{Chapter }{ParentGroup (for IsRegularRootedTreeGroup)}}
\logpage{[ 1, 1, 5 ]}\nobreak
\hyperdef{L}{X7943CE9B7AB931AC}{}
{\noindent\textcolor{FuncColor}{$\triangleright$\enspace\texttt{ParentGroup({\mdseries\slshape G})\index{ParentGroup@\texttt{ParentGroup}!for IsRegularRootedTreeGroup}
\label{ParentGroup:for IsRegularRootedTreeGroup}
}\hfill{\scriptsize (attribute)}}\\
\textbf{\indent Returns:\ }
 the regular rooted tree group that arises from \mbox{\texttt{\mdseries\slshape G}} by restricting to $T_{k,n-1}$. 



 The argument of this attribute is a regular rooted tree group \mbox{\texttt{\mdseries\slshape G}} $\le\mathrm{Aut}(T_{k,k})$ (\texttt{IsRegularRootedTreeGroup} (\ref{IsRegularRootedTreeGroup})). 

 }

 

 
\begin{Verbatim}[commandchars=!@|,fontsize=\small,frame=single,label=Example]
  !gapprompt@gap>| !gapinput@G:=AutT(2,4);|
  <permutation group of size 32768 with 15 generators>
  !gapprompt@gap>| !gapinput@ParentGroup(G)=AutT(2,3);|
  true
\end{Verbatim}
 

\subsection{\textcolor{Chapter }{IsSelfReplicating (for IsRegularRootedTreeGroup)}}
\logpage{[ 1, 1, 6 ]}\nobreak
\hyperdef{L}{X854CB0D683905369}{}
{\noindent\textcolor{FuncColor}{$\triangleright$\enspace\texttt{IsSelfReplicating({\mdseries\slshape G})\index{IsSelfReplicating@\texttt{IsSelfReplicating}!for IsRegularRootedTreeGroup}
\label{IsSelfReplicating:for IsRegularRootedTreeGroup}
}\hfill{\scriptsize (property)}}\\
\textbf{\indent Returns:\ }
 \texttt{true}, if \mbox{\texttt{\mdseries\slshape G}} is self-replicating, and \texttt{false} otherwise. 



 The argument of this attribute is a regular rooted tree group \mbox{\texttt{\mdseries\slshape G}} $\le\mathrm{Aut}(T_{k,k})$ (\texttt{IsRegularRootedTreeGroup} (\ref{IsRegularRootedTreeGroup})). 

 }

 

 
\begin{Verbatim}[commandchars=!@|,fontsize=\small,frame=single,label=Example]
  !gapprompt@gap>| !gapinput@G:=AutT(2,2);|
  Group([ (1,2), (3,4), (1,3)(2,4) ])
  !gapprompt@gap>| !gapinput@subgroups:=AllSubgroups(G);;|
  !gapprompt@gap>| !gapinput@Apply(subgroups,H->RegularRootedTreeGroup(2,2,H));|
  !gapprompt@gap>| !gapinput@for H in subgroups do Print(IsSelfReplicating(H),"\n"); od;|
  false
  false
  false
  false
  false
  false
  false
  true
  true
  true
\end{Verbatim}
 

\subsection{\textcolor{Chapter }{HasSufficientRigidAutomorphisms (for IsRegularRootedTreeGroup)}}
\logpage{[ 1, 1, 7 ]}\nobreak
\hyperdef{L}{X84DDFFAE8701C3F8}{}
{\noindent\textcolor{FuncColor}{$\triangleright$\enspace\texttt{HasSufficientRigidAutomorphisms({\mdseries\slshape G})\index{HasSufficientRigidAutomorphisms@\texttt{HasSufficientRigidAutomorphisms}!for IsRegularRootedTreeGroup}
\label{HasSufficientRigidAutomorphisms:for IsRegularRootedTreeGroup}
}\hfill{\scriptsize (property)}}\\
\textbf{\indent Returns:\ }
 \texttt{true}, if \mbox{\texttt{\mdseries\slshape G}} has sufficient rigid automorphisms, and \texttt{false} otherwise. 



 The argument of this attribute is a regular rooted tree group \mbox{\texttt{\mdseries\slshape G}} $\le\mathrm{Aut}(T_{k,k})$ (\texttt{IsRegularRootedTreeGroup} (\ref{IsRegularRootedTreeGroup})). 

 }

 

 
\begin{Verbatim}[commandchars=!@|,fontsize=\small,frame=single,label=Example]
  to do
\end{Verbatim}
 

\subsection{\textcolor{Chapter }{RepresentativeWithSufficientRigidAutomorphisms (for IsRegularRootedTreeGroup)}}
\logpage{[ 1, 1, 8 ]}\nobreak
\hyperdef{L}{X7D06D7AF7B938F4A}{}
{\noindent\textcolor{FuncColor}{$\triangleright$\enspace\texttt{RepresentativeWithSufficientRigidAutomorphisms({\mdseries\slshape G})\index{RepresentativeWithSufficientRigidAutomorphisms@\texttt{Representative}\-\texttt{With}\-\texttt{Sufficient}\-\texttt{Rigid}\-\texttt{Automorphisms}!for IsRegularRootedTreeGroup}
\label{RepresentativeWithSufficientRigidAutomorphisms:for IsRegularRootedTreeGroup}
}\hfill{\scriptsize (attribute)}}\\
\textbf{\indent Returns:\ }
 a regular rooted tree group which is conjugate to \mbox{\texttt{\mdseries\slshape G}} in $\mathrm{Aut}(T_{k,n})$ and which has sufficient rigid automorphisms, i.e. it satisfies \texttt{HasSufficientRigidAutomorphisms} (\ref{HasSufficientRigidAutomorphisms}). This returned group is \mbox{\texttt{\mdseries\slshape G}} itself, if \mbox{\texttt{\mdseries\slshape G}} already has sufficient rigid automorphisms. Furthermore, the returned group
has the same parent group as \mbox{\texttt{\mdseries\slshape G}} if the parent group of \mbox{\texttt{\mdseries\slshape G}} has sufficient rigid automorphisms. 



 The argument of this attribute is a regular rooted tree group \mbox{\texttt{\mdseries\slshape G}} $\le\mathrm{Aut}(T_{k,k})$ (\texttt{IsRegularRootedTreeGroup} (\ref{IsRegularRootedTreeGroup})), which is self-replicating (\texttt{IsSelfReplicating} (\ref{IsSelfReplicating})). 

 }

 

 
\begin{Verbatim}[commandchars=!@|,fontsize=\small,frame=single,label=Example]
  to do
\end{Verbatim}
 

\subsection{\textcolor{Chapter }{MaximalExtension (for IsRegularRootedTreeGroup)}}
\logpage{[ 1, 1, 9 ]}\nobreak
\hyperdef{L}{X7D9FB6D5854A17BE}{}
{\noindent\textcolor{FuncColor}{$\triangleright$\enspace\texttt{MaximalExtension({\mdseries\slshape G})\index{MaximalExtension@\texttt{MaximalExtension}!for IsRegularRootedTreeGroup}
\label{MaximalExtension:for IsRegularRootedTreeGroup}
}\hfill{\scriptsize (attribute)}}\\
\textbf{\indent Returns:\ }
 the regular rooted tree group $M($\mbox{\texttt{\mdseries\slshape G}}$)\le\mathrm{Aut}(T_{k,n})$ which is the unique maximal self-replicating extension of \mbox{\texttt{\mdseries\slshape G}} to $T_{k,n+1}$. 



 The argument of this attribute is a regular rooted tree group \mbox{\texttt{\mdseries\slshape G}} $\le\mathrm{Aut}(T_{k,k})$ (\texttt{IsRegularRootedTreeGroup} (\ref{IsRegularRootedTreeGroup})), which is self-replicating (\texttt{IsSelfReplicating} (\ref{IsSelfReplicating})) and has sufficient rigid automorphisms (\texttt{HasSufficientRigidAutomorphisms} (\ref{HasSufficientRigidAutomorphisms})). 

 }

 

 
\begin{Verbatim}[commandchars=!@|,fontsize=\small,frame=single,label=Example]
  to do
\end{Verbatim}
 

\subsection{\textcolor{Chapter }{ConjugacyClassRepsSelfReplicatingGroupsWithProjection (for IsRegularRootedTreeGroup)}}
\logpage{[ 1, 1, 10 ]}\nobreak
\hyperdef{L}{X876FC3387B6E7183}{}
{\noindent\textcolor{FuncColor}{$\triangleright$\enspace\texttt{ConjugacyClassRepsSelfReplicatingGroupsWithProjection({\mdseries\slshape G})\index{ConjugacyClassRepsSelfReplicatingGroupsWithProjection@\texttt{Conjugacy}\-\texttt{Class}\-\texttt{Reps}\-\texttt{Self}\-\texttt{Replicating}\-\texttt{Groups}\-\texttt{With}\-\texttt{Projection}!for IsRegularRootedTreeGroup}
\label{ConjugacyClassRepsSelfReplicatingGroupsWithProjection:for IsRegularRootedTreeGroup}
}\hfill{\scriptsize (attribute)}}\\
\textbf{\indent Returns:\ }
 a list $\mathrm{Aut}(T_{k,n+1}$-conjugacy class representatives of regular rooted tree groups which are
self-replicating, have sufficient rigid automorphisms and whose parent group
is \mbox{\texttt{\mdseries\slshape G}}. 



 The argument of this attribute is a regular rooted tree group \mbox{\texttt{\mdseries\slshape G}} $\le\mathrm{Aut}(T_{k,k})$ (\texttt{IsRegularRootedTreeGroup} (\ref{IsRegularRootedTreeGroup})), which is self-replicating (\texttt{IsSelfReplicating} (\ref{IsSelfReplicating})) and has sufficient rigid automorphisms (\texttt{HasSufficientRigidAutomorphisms} (\ref{HasSufficientRigidAutomorphisms})). 

 }

 

 
\begin{Verbatim}[commandchars=!@|,fontsize=\small,frame=single,label=Example]
  to do
\end{Verbatim}
 

\subsection{\textcolor{Chapter }{ConjugacyClassRepsSelfReplicatingGroupsWithConjugateProjection (for IsRegularRootedTreeGroup)}}
\logpage{[ 1, 1, 11 ]}\nobreak
\hyperdef{L}{X7D7A396F7F9DF4D8}{}
{\noindent\textcolor{FuncColor}{$\triangleright$\enspace\texttt{ConjugacyClassRepsSelfReplicatingGroupsWithConjugateProjection({\mdseries\slshape G})\index{ConjugacyClassRepsSelfReplicatingGroupsWithConjugateProjection@\texttt{Conjugacy}\-\texttt{Class}\-\texttt{Reps}\-\texttt{Self}\-\texttt{Replicating}\-\texttt{Groups}\-\texttt{With}\-\texttt{Conjugate}\-\texttt{Projection}!for IsRegularRootedTreeGroup}
\label{ConjugacyClassRepsSelfReplicatingGroupsWithConjugateProjection:for IsRegularRootedTreeGroup}
}\hfill{\scriptsize (attribute)}}\\
\textbf{\indent Returns:\ }
 a list $\mathrm{Aut}(T_{k,n+1}$-conjugacy class representatives of regular rooted tree groups which are
self-replicating, have sufficient rigid automorphisms and whose parent group
is conjugate to \mbox{\texttt{\mdseries\slshape G}}. 



 The argument of this attribute is a regular rooted tree group \mbox{\texttt{\mdseries\slshape G}} $\le\mathrm{Aut}(T_{k,k})$ (\texttt{IsRegularRootedTreeGroup} (\ref{IsRegularRootedTreeGroup})), which is self-replicating (\texttt{IsSelfReplicating} (\ref{IsSelfReplicating})) and has sufficient rigid automorphisms (\texttt{HasSufficientRigidAutomorphisms} (\ref{HasSufficientRigidAutomorphisms})). 

 }

 

 
\begin{Verbatim}[commandchars=!@|,fontsize=\small,frame=single,label=Example]
  to do
\end{Verbatim}
 }

 
\section{\textcolor{Chapter }{Auxiliary methods}}\label{Chapter_The_package_Section_Auxiliary_methods}
\logpage{[ 1, 2, 0 ]}
\hyperdef{L}{X84EE59EB82D1AB79}{}
{
  This section explains the methods of this package. 

\subsection{\textcolor{Chapter }{RemoveConjugates}}
\logpage{[ 1, 2, 1 ]}\nobreak
\hyperdef{L}{X795ACCF4858A8ED2}{}
{\noindent\textcolor{FuncColor}{$\triangleright$\enspace\texttt{RemoveConjugates({\mdseries\slshape G, subgroups})\index{RemoveConjugates@\texttt{RemoveConjugates}}
\label{RemoveConjugates}
}\hfill{\scriptsize (function)}}\\
\textbf{\indent Returns:\ }
 n/a. This method removes \mbox{\texttt{\mdseries\slshape G}}-conjugates from the mutable list \mbox{\texttt{\mdseries\slshape subgroups}}. 



 The arguments of this method are a group \mbox{\texttt{\mdseries\slshape G}} and a mutable list \mbox{\texttt{\mdseries\slshape subgroups}} of subgroups of $G$. 

 }

 

 
\begin{Verbatim}[commandchars=!@|,fontsize=\small,frame=single,label=Example]
  !gapprompt@gap>| !gapinput@G:=SymmetricGroup(3);|
  Sym( [ 1 .. 3 ] )
  !gapprompt@gap>| !gapinput@subgroups:=[Group((1,2)),Group((2,3))];|
  [ Group([ (1,2) ]), Group([ (2,3) ]) ]
  !gapprompt@gap>| !gapinput@RemoveConjugates(G,subgroups);|
  !gapprompt@gap>| !gapinput@subgroups;|
  [ Group([ (1,2) ]) ]
\end{Verbatim}
 

\subsection{\textcolor{Chapter }{AutT}}
\logpage{[ 1, 2, 2 ]}\nobreak
\hyperdef{L}{X843B26BC8517173F}{}
{\noindent\textcolor{FuncColor}{$\triangleright$\enspace\texttt{AutT({\mdseries\slshape k, n})\index{AutT@\texttt{AutT}}
\label{AutT}
}\hfill{\scriptsize (function)}}\\
\textbf{\indent Returns:\ }
 the regular rooted tree group $\mathrm{Aut}(T_{k,n})$ (\texttt{IsRegularRootedTreeGroup} (\ref{IsRegularRootedTreeGroup})) as a permutation group of the $k^{n}$ leaves of $T_{k,n}$. 



 The arguments of this method are a degree \mbox{\texttt{\mdseries\slshape k}} $\in\mathbb{N}_{\ge 2}$ and a depth \mbox{\texttt{\mdseries\slshape n}} $\in\mathbb{N}$. 

 }

 

 
\begin{Verbatim}[commandchars=!@|,fontsize=\small,frame=single,label=Example]
  !gapprompt@gap>| !gapinput@G:=AutT(2,2);|
  Group([ (1,2), (3,4), (1,3)(2,4) ])
  !gapprompt@gap>| !gapinput@RegularRootedTreeGroupDegree(G);|
  2
  !gapprompt@gap>| !gapinput@RegularRootedTreeGroupDepth(G);|
  2
\end{Verbatim}
 

\subsection{\textcolor{Chapter }{BelowAction}}
\logpage{[ 1, 2, 3 ]}\nobreak
\hyperdef{L}{X7EE8301C7EC8654B}{}
{\noindent\textcolor{FuncColor}{$\triangleright$\enspace\texttt{BelowAction({\mdseries\slshape k, n, aut, i})\index{BelowAction@\texttt{BelowAction}}
\label{BelowAction}
}\hfill{\scriptsize (function)}}\\
\textbf{\indent Returns:\ }
 the regular rooted tree group $\mathrm{Aut}(T_{k,n})$ as a permutation group of the $k^{n}$ leaves of $T_{k,n}$. 



 The arguments of this method are a degree \mbox{\texttt{\mdseries\slshape k}} $\in\mathbb{N}_{\ge 2}$ and a depth \mbox{\texttt{\mdseries\slshape n}} $\in\mathbb{N}$. 

 }

 
\begin{Verbatim}[commandchars=!@|,fontsize=\small,frame=single,label=Example]
  !gapprompt@gap>| !gapinput@G:=AutT(2,2);|
  Group([ (1,2), (3,4), (1,3)(2,4) ])
  !gapprompt@gap>| !gapinput@a:=Random(G);|
  (1,3,2,4)
  !gapprompt@gap>| !gapinput@BelowAction(2,2,a,1);|
  ()
  !gapprompt@gap>| !gapinput@BelowAction(2,2,a,2);|
  (1,2)
\end{Verbatim}
 }

 }

   
\chapter{\textcolor{Chapter }{The library}}\label{Chapter_The_library}
\logpage{[ 2, 0, 0 ]}
\hyperdef{L}{X7E20D66B81464B38}{}
{
  
\section{\textcolor{Chapter }{Methods}}\label{Chapter_The_library_Section_Methods}
\logpage{[ 2, 1, 0 ]}
\hyperdef{L}{X8606FDCE878850EF}{}
{
  

\subsection{\textcolor{Chapter }{bar (for IsObject)}}
\logpage{[ 2, 1, 1 ]}\nobreak
\hyperdef{L}{X7E429C927AD46053}{}
{\noindent\textcolor{FuncColor}{$\triangleright$\enspace\texttt{bar({\mdseries\slshape arg})\index{bar@\texttt{bar}!for IsObject}
\label{bar:for IsObject}
}\hfill{\scriptsize (filter)}}\\
\textbf{\indent Returns:\ }
\texttt{true} or \texttt{false} 



 foo 

 }

 }

 }

 \def\indexname{Index\logpage{[ "Ind", 0, 0 ]}
\hyperdef{L}{X83A0356F839C696F}{}
}

\cleardoublepage
\phantomsection
\addcontentsline{toc}{chapter}{Index}


\printindex

\immediate\write\pagenrlog{["Ind", 0, 0], \arabic{page},}
\immediate\write\pagenrlog{["Ind", 0, 0], \arabic{page},}
\immediate\write\pagenrlog{["Ind", 0, 0], \arabic{page},}
\immediate\write\pagenrlog{["Ind", 0, 0], \arabic{page},}
\immediate\write\pagenrlog{["Ind", 0, 0], \arabic{page},}
\immediate\write\pagenrlog{["Ind", 0, 0], \arabic{page},}
\immediate\write\pagenrlog{["Ind", 0, 0], \arabic{page},}
\immediate\write\pagenrlog{["Ind", 0, 0], \arabic{page},}
\immediate\write\pagenrlog{["Ind", 0, 0], \arabic{page},}
\immediate\write\pagenrlog{["Ind", 0, 0], \arabic{page},}
\immediate\write\pagenrlog{["Ind", 0, 0], \arabic{page},}
\newpage
\immediate\write\pagenrlog{["End"], \arabic{page}];}
\immediate\closeout\pagenrlog
\end{document}
