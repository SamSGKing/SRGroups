% generated by GAPDoc2LaTeX from XML source (Frank Luebeck)
\documentclass[a4paper,11pt]{report}

\usepackage[top=37mm,bottom=37mm,left=27mm,right=27mm]{geometry}
\sloppy
\pagestyle{myheadings}
\usepackage{amssymb}
\usepackage[utf8]{inputenc}
\usepackage{makeidx}
\makeindex
\usepackage{color}
\definecolor{FireBrick}{rgb}{0.5812,0.0074,0.0083}
\definecolor{RoyalBlue}{rgb}{0.0236,0.0894,0.6179}
\definecolor{RoyalGreen}{rgb}{0.0236,0.6179,0.0894}
\definecolor{RoyalRed}{rgb}{0.6179,0.0236,0.0894}
\definecolor{LightBlue}{rgb}{0.8544,0.9511,1.0000}
\definecolor{Black}{rgb}{0.0,0.0,0.0}

\definecolor{linkColor}{rgb}{0.0,0.0,0.554}
\definecolor{citeColor}{rgb}{0.0,0.0,0.554}
\definecolor{fileColor}{rgb}{0.0,0.0,0.554}
\definecolor{urlColor}{rgb}{0.0,0.0,0.554}
\definecolor{promptColor}{rgb}{0.0,0.0,0.589}
\definecolor{brkpromptColor}{rgb}{0.589,0.0,0.0}
\definecolor{gapinputColor}{rgb}{0.589,0.0,0.0}
\definecolor{gapoutputColor}{rgb}{0.0,0.0,0.0}

%%  for a long time these were red and blue by default,
%%  now black, but keep variables to overwrite
\definecolor{FuncColor}{rgb}{0.0,0.0,0.0}
%% strange name because of pdflatex bug:
\definecolor{Chapter }{rgb}{0.0,0.0,0.0}
\definecolor{DarkOlive}{rgb}{0.1047,0.2412,0.0064}


\usepackage{fancyvrb}

\usepackage{mathptmx,helvet}
\usepackage[T1]{fontenc}
\usepackage{textcomp}


\usepackage[
            pdftex=true,
            bookmarks=true,        
            a4paper=true,
            pdftitle={Written with GAPDoc},
            pdfcreator={LaTeX with hyperref package / GAPDoc},
            colorlinks=true,
            backref=page,
            breaklinks=true,
            linkcolor=linkColor,
            citecolor=citeColor,
            filecolor=fileColor,
            urlcolor=urlColor,
            pdfpagemode={UseNone}, 
           ]{hyperref}

\newcommand{\maintitlesize}{\fontsize{50}{55}\selectfont}

% write page numbers to a .pnr log file for online help
\newwrite\pagenrlog
\immediate\openout\pagenrlog =\jobname.pnr
\immediate\write\pagenrlog{PAGENRS := [}
\newcommand{\logpage}[1]{\protect\write\pagenrlog{#1, \thepage,}}
%% were never documented, give conflicts with some additional packages

\newcommand{\GAP}{\textsf{GAP}}

%% nicer description environments, allows long labels
\usepackage{enumitem}
\setdescription{style=nextline}

%% depth of toc
\setcounter{tocdepth}{1}





%% command for ColorPrompt style examples
\newcommand{\gapprompt}[1]{\color{promptColor}{\bfseries #1}}
\newcommand{\gapbrkprompt}[1]{\color{brkpromptColor}{\bfseries #1}}
\newcommand{\gapinput}[1]{\color{gapinputColor}{#1}}


\begin{document}

\logpage{[ 0, 0, 0 ]}
\begin{titlepage}
\mbox{}\vfill

\begin{center}{\maintitlesize \textbf{ SRGroups: Self-replicating groups of regular rooted trees. \mbox{}}}\\
\vfill

\hypersetup{pdftitle= SRGroups: Self-replicating groups of regular rooted trees. }
\markright{\scriptsize \mbox{}\hfill  SRGroups: Self-replicating groups of regular rooted trees.  \hfill\mbox{}}
{\Huge \textbf{ Self-replicating groups of regular rooted trees. \mbox{}}}\\
\vfill

{\Huge  0.1 \mbox{}}\\[1cm]
{ 17 December 2020 \mbox{}}\\[1cm]
\mbox{}\\[2cm]
{\Large \textbf{ Sam King\\
   \mbox{}}}\\
{\Large \textbf{ Sarah Shotter\\
   \mbox{}}}\\
\hypersetup{pdfauthor= Sam King\\
   ;  Sarah Shotter\\
   }
\end{center}\vfill

\mbox{}\\
{\mbox{}\\
\small \noindent \textbf{ Sam King\\
   }  Email: \href{mailto://sam.king@newcastle.edu.au} {\texttt{sam.king@newcastle.edu.au}}\\
  Address: \begin{minipage}[t]{8cm}\noindent
 University Drive, Callaghan NSW 2308\\
 \end{minipage}
}\\
{\mbox{}\\
\small \noindent \textbf{ Sarah Shotter\\
   }  Email: \href{mailto://sarah.shotter@newcastle.edu.au} {\texttt{sarah.shotter@newcastle.edu.au}}\\
  Address: \begin{minipage}[t]{8cm}\noindent
 University Drive, Callaghan NSW 2308\\
 \end{minipage}
}\\
\end{titlepage}

\newpage\setcounter{page}{2}
{\small 
\section*{Abstract}
\logpage{[ 0, 0, 1 ]}
 To do. \mbox{}}\\[1cm]
{\small 
\section*{Copyright}
\logpage{[ 0, 0, 2 ]}
 \textsf{SRGroups} is free software; you can redistribute it and/or modify it under the terms of
the \href{http://www.fsf.org/licenses/gpl.html} {GNU General Public License} as published by the Free Software Foundation; either version 3 of the License,
or (at your option) any later version. \mbox{}}\\[1cm]
{\small 
\section*{Acknowledgements}
\logpage{[ 0, 0, 3 ]}
 DE210100180, FL170100032. \mbox{}}\\[1cm]
\newpage

\def\contentsname{Contents\logpage{[ 0, 0, 4 ]}}

\tableofcontents
\newpage

     
\chapter{\textcolor{Chapter }{Introduction}}\label{Chapter_Introduction}
\logpage{[ 1, 0, 0 ]}
\hyperdef{L}{X7DFB63A97E67C0A1}{}
{
  SRGroups is a package which does some interesting and cool things. To be
continued... }

   
\chapter{\textcolor{Chapter }{Functionality}}\label{Chapter_Functionality}
\logpage{[ 2, 0, 0 ]}
\hyperdef{L}{X87F1120883F5B4D0}{}
{
  
\section{\textcolor{Chapter }{Methods}}\label{Chapter_Functionality_Section_Methods}
\logpage{[ 2, 1, 0 ]}
\hyperdef{L}{X8606FDCE878850EF}{}
{
  

\subsection{\textcolor{Chapter }{IsRegularRootedTreeGroup (for IsPermGroup)}}
\logpage{[ 2, 1, 1 ]}\nobreak
\hyperdef{L}{X81F8437A7C2E5C87}{}
{\noindent\textcolor{FuncColor}{$\triangleright$\enspace\texttt{IsRegularRootedTreeGroup({\mdseries\slshape arg})\index{IsRegularRootedTreeGroup@\texttt{IsRegularRootedTreeGroup}!for IsPermGroup}
\label{IsRegularRootedTreeGroup:for IsPermGroup}
}\hfill{\scriptsize (filter)}}\\
\textbf{\indent Returns:\ }
\texttt{true} or \texttt{false} 



 Checks whether the input group is a regular-rooted tree group. }

 

\subsection{\textcolor{Chapter }{RegularRootedTreeGroup (for IsInt, IsInt, IsPermGroup)}}
\logpage{[ 2, 1, 2 ]}\nobreak
\hyperdef{L}{X7F279D957D29554A}{}
{\noindent\textcolor{FuncColor}{$\triangleright$\enspace\texttt{RegularRootedTreeGroup({\mdseries\slshape arg1, arg2, arg3})\index{RegularRootedTreeGroup@\texttt{RegularRootedTreeGroup}!for IsInt, IsInt, IsPermGroup}
\label{RegularRootedTreeGroup:for IsInt, IsInt, IsPermGroup}
}\hfill{\scriptsize (operation)}}\\


 Creates a regular-rooted tree group with attributes \mbox{\texttt{\mdseries\slshape RegularRootedTreeGroupDegree}} and \mbox{\texttt{\mdseries\slshape RegularRootedTreeGroupDepth}}. }

 }

 
\section{\textcolor{Chapter }{Functions}}\label{Chapter_Functionality_Section_Functions}
\logpage{[ 2, 2, 0 ]}
\hyperdef{L}{X86FA580F8055B274}{}
{
  

\subsection{\textcolor{Chapter }{AllSRGroups}}
\logpage{[ 2, 2, 1 ]}\nobreak
\hyperdef{L}{X845F9F227B97062C}{}
{\noindent\textcolor{FuncColor}{$\triangleright$\enspace\texttt{AllSRGroups({\mdseries\slshape arg})\index{AllSRGroups@\texttt{AllSRGroups}}
\label{AllSRGroups}
}\hfill{\scriptsize (function)}}\\


 Main library search function. Has several possible input arguments such as \mbox{\texttt{\mdseries\slshape Degree}}, \mbox{\texttt{\mdseries\slshape Level}} (or \mbox{\texttt{\mdseries\slshape Depth}}), \mbox{\texttt{\mdseries\slshape Number}}, \mbox{\texttt{\mdseries\slshape Projection}}, \mbox{\texttt{\mdseries\slshape Subgroup}}, \mbox{\texttt{\mdseries\slshape Size}}, \mbox{\texttt{\mdseries\slshape NumberOfGenerators}}, and \mbox{\texttt{\mdseries\slshape IsAbelian}}. }

 
\begin{Verbatim}[commandchars=!@|,fontsize=\small,frame=single,label=Example]
  !gapprompt@gap>| !gapinput@AllSRGroups(Degree, 2, Level, 4, IsAbelian, true);|
  [ SRGroup(2,4,2), SRGroup(2,4,9), SRGroup(2,4,12), SRGroup(2,4,14) ]
  !gapprompt@gap>| !gapinput@Size(last[1]);|
  16
  !gapprompt@gap>| !gapinput@AllSRGroups(Degree,2,Level,4,NumberOfGenerators,4);|
  [ SRGroup(2,4,11), SRGroup(2,4,12), SRGroup(2,4,16), SRGroup(2,4,20), SRGroup(2,4,23), SRGroup(2,4,24),
   SRGroup(2,4,25), SRGroup(2,4,26), SRGroup(2,4,40), SRGroup(2,4,43), SRGroup(2,4,46), SRGroup(2,4,47),
   SRGroup(2,4,50), SRGroup(2,4,66), SRGroup(2,4,70), SRGroup(2,4,71), SRGroup(2,4,72), SRGroup(2,4,73),
   SRGroup(2,4,74), SRGroup(2,4,75), SRGroup(2,4,76), SRGroup(2,4,84), SRGroup(2,4,90), SRGroup(2,4,91),
   SRGroup(2,4,93), SRGroup(2,4,95), SRGroup(2,4,97), SRGroup(2,4,102), SRGroup(2,4,108) ]
\end{Verbatim}
 

\subsection{\textcolor{Chapter }{SRGroupsInfo}}
\logpage{[ 2, 2, 2 ]}\nobreak
\hyperdef{L}{X7A191D0B85ABBB01}{}
{\noindent\textcolor{FuncColor}{$\triangleright$\enspace\texttt{SRGroupsInfo({\mdseries\slshape arg})\index{SRGroupsInfo@\texttt{SRGroupsInfo}}
\label{SRGroupsInfo}
}\hfill{\scriptsize (function)}}\\


 Insert documentation for your function here }

 

\subsection{\textcolor{Chapter }{AllSRGroupsInfo}}
\logpage{[ 2, 2, 3 ]}\nobreak
\hyperdef{L}{X808A9F918166020B}{}
{\noindent\textcolor{FuncColor}{$\triangleright$\enspace\texttt{AllSRGroupsInfo({\mdseries\slshape arg})\index{AllSRGroupsInfo@\texttt{AllSRGroupsInfo}}
\label{AllSRGroupsInfo}
}\hfill{\scriptsize (function)}}\\


 Works the same as the main library search function \mbox{\texttt{\mdseries\slshape AllSRGroups}}, except returns useful information about the group(s) in list form: [\mbox{\texttt{\mdseries\slshape Generators}}, \mbox{\texttt{\mdseries\slshape Name}}, \mbox{\texttt{\mdseries\slshape Parent Name}}, \mbox{\texttt{\mdseries\slshape Children Names}}]. }

 
\begin{Verbatim}[commandchars=!@|,fontsize=\small,frame=single,label=Example]
  !gapprompt@gap>| !gapinput@AllSRGroupsInfo(Degree, 2, Level, 3, IsAbelian, true);|
  [ [ [ (1,5,4,8,2,6,3,7), (1,4,2,3)(5,8,6,7), (1,2)(3,4)(5,6)(7,8) ], "SRGroup(2,3,1)", "SRGroup(2,2,1)", [ "SRGroup(2,4,1)", "SRGroup(2,4,2)" ] ],
  [ [ (1,5,2,6)(3,7,4,8), (1,3)(2,4)(5,7)(6,8), (1,2)(3,4)(5,6)(7,8) ], "SRGroup(2,3,4)", "SRGroup(2,2,2)", [ "SRGroup(2,4,8)", "SRGroup(2,4,9)", "SRGroup(2,4,10)" ] ], 
  [ [ (1,3)(2,4)(5,7)(6,8), (1,5)(2,6)(3,7)(4,8), (1,2)(3,4)(5,6)(7,8) ], "SRGroup(2,3,5)", "SRGroup(2,2,2)", [ "SRGroup(2,4,11)", "SRGroup(2,4,12)", "SRGroup(2,4,13)", "SRGroup(2,4,14)", "SRGroup(2,4,15)" ] ] ]
\end{Verbatim}
 

\subsection{\textcolor{Chapter }{CheckSRProjections}}
\logpage{[ 2, 2, 4 ]}\nobreak
\hyperdef{L}{X7DC9A60E87E350AF}{}
{\noindent\textcolor{FuncColor}{$\triangleright$\enspace\texttt{CheckSRProjections({\mdseries\slshape arg})\index{CheckSRProjections@\texttt{CheckSRProjections}}
\label{CheckSRProjections}
}\hfill{\scriptsize (function)}}\\


 

 }

 

\subsection{\textcolor{Chapter }{StringVariables}}
\logpage{[ 2, 2, 5 ]}\nobreak
\hyperdef{L}{X7E515BA882F73045}{}
{\noindent\textcolor{FuncColor}{$\triangleright$\enspace\texttt{StringVariables({\mdseries\slshape arg})\index{StringVariables@\texttt{StringVariables}}
\label{StringVariables}
}\hfill{\scriptsize (function)}}\\


 

 }

 

\subsection{\textcolor{Chapter }{UnbindVariables}}
\logpage{[ 2, 2, 6 ]}\nobreak
\hyperdef{L}{X82C5A9557EC8F8A6}{}
{\noindent\textcolor{FuncColor}{$\triangleright$\enspace\texttt{UnbindVariables({\mdseries\slshape arg})\index{UnbindVariables@\texttt{UnbindVariables}}
\label{UnbindVariables}
}\hfill{\scriptsize (function)}}\\


 

 }

 

\subsection{\textcolor{Chapter }{SRDegrees}}
\logpage{[ 2, 2, 7 ]}\nobreak
\hyperdef{L}{X7BF255357B870D37}{}
{\noindent\textcolor{FuncColor}{$\triangleright$\enspace\texttt{SRDegrees({\mdseries\slshape arg})\index{SRDegrees@\texttt{SRDegrees}}
\label{SRDegrees}
}\hfill{\scriptsize (function)}}\\


 Returns all of the degrees currently stored in the SRGroups library. }

 
\begin{Verbatim}[commandchars=!@|,fontsize=\small,frame=single,label=Example]
  !gapprompt@gap>| !gapinput@SRDegrees();|
  [ 2, 2, 2, 2, 3, 3, 3, 4, 4, 5, 6, 7, 8, 9, 10, 11, 12, 13, 14, 15, 16 ]
\end{Verbatim}
 

\subsection{\textcolor{Chapter }{SRLevels}}
\logpage{[ 2, 2, 8 ]}\nobreak
\hyperdef{L}{X7CDBB45386329E21}{}
{\noindent\textcolor{FuncColor}{$\triangleright$\enspace\texttt{SRLevels({\mdseries\slshape arg})\index{SRLevels@\texttt{SRLevels}}
\label{SRLevels}
}\hfill{\scriptsize (function)}}\\


 Returns all of the levels currently stored in the SRGroups library for an
input RegularRootedTreeGroupDegree, \mbox{\texttt{\mdseries\slshape deg}}. }

 
\begin{Verbatim}[commandchars=!@|,fontsize=\small,frame=single,label=Example]
  !gapprompt@gap>| !gapinput@SRLevels(2);|
  [ 1, 2, 3, 4 ]
\end{Verbatim}
 

\subsection{\textcolor{Chapter }{AutT}}
\logpage{[ 2, 2, 9 ]}\nobreak
\hyperdef{L}{X843B26BC8517173F}{}
{\noindent\textcolor{FuncColor}{$\triangleright$\enspace\texttt{AutT({\mdseries\slshape k, n})\index{AutT@\texttt{AutT}}
\label{AutT}
}\hfill{\scriptsize (function)}}\\
\textbf{\indent Returns:\ }
 the regular rooted tree group $\mathrm{Aut}(T_{k,n})$ as a permutation group of the $k^{n}$ leaves of $T_{k,n}$. 



 The arguments of this method are a degree \mbox{\texttt{\mdseries\slshape k}} $\in\mathbb{N}_{\ge 2}$ and a depth \mbox{\texttt{\mdseries\slshape n}} $\in\mathbb{N}$. 

 }

 

 
\begin{Verbatim}[commandchars=!@|,fontsize=\small,frame=single,label=Example]
  !gapprompt@gap>| !gapinput@G:=AutT(2,2);|
  Group([ (1,2), (3,4), (1,3)(2,4) ])
  !gapprompt@gap>| !gapinput@Size(G);|
  8
\end{Verbatim}
 }

 }

 \def\indexname{Index\logpage{[ "Ind", 0, 0 ]}
\hyperdef{L}{X83A0356F839C696F}{}
}

\cleardoublepage
\phantomsection
\addcontentsline{toc}{chapter}{Index}


\printindex

\immediate\write\pagenrlog{["Ind", 0, 0], \arabic{page},}
\immediate\write\pagenrlog{["Ind", 0, 0], \arabic{page},}
\immediate\write\pagenrlog{["Ind", 0, 0], \arabic{page},}
\immediate\write\pagenrlog{["Ind", 0, 0], \arabic{page},}
\immediate\write\pagenrlog{["Ind", 0, 0], \arabic{page},}
\immediate\write\pagenrlog{["Ind", 0, 0], \arabic{page},}
\immediate\write\pagenrlog{["Ind", 0, 0], \arabic{page},}
\immediate\write\pagenrlog{["Ind", 0, 0], \arabic{page},}
\immediate\write\pagenrlog{["Ind", 0, 0], \arabic{page},}
\immediate\write\pagenrlog{["Ind", 0, 0], \arabic{page},}
\immediate\write\pagenrlog{["Ind", 0, 0], \arabic{page},}
\immediate\write\pagenrlog{["Ind", 0, 0], \arabic{page},}
\immediate\write\pagenrlog{["Ind", 0, 0], \arabic{page},}
\immediate\write\pagenrlog{["Ind", 0, 0], \arabic{page},}
\immediate\write\pagenrlog{["Ind", 0, 0], \arabic{page},}
\immediate\write\pagenrlog{["Ind", 0, 0], \arabic{page},}
\immediate\write\pagenrlog{["Ind", 0, 0], \arabic{page},}
\immediate\write\pagenrlog{["Ind", 0, 0], \arabic{page},}
\immediate\write\pagenrlog{["Ind", 0, 0], \arabic{page},}
\immediate\write\pagenrlog{["Ind", 0, 0], \arabic{page},}
\immediate\write\pagenrlog{["Ind", 0, 0], \arabic{page},}
\immediate\write\pagenrlog{["Ind", 0, 0], \arabic{page},}
\immediate\write\pagenrlog{["Ind", 0, 0], \arabic{page},}
\immediate\write\pagenrlog{["Ind", 0, 0], \arabic{page},}
\immediate\write\pagenrlog{["Ind", 0, 0], \arabic{page},}
\immediate\write\pagenrlog{["Ind", 0, 0], \arabic{page},}
\newpage
\immediate\write\pagenrlog{["End"], \arabic{page}];}
\immediate\closeout\pagenrlog
\end{document}
